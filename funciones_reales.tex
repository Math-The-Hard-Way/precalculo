\section{Funciones reales}

\subsection{Definición}
{}
  Una \emph{función} $f$ es una regla que asigna a cada elemento $x$ de un conjunto $A$ un único elemento $y$ de otro conjunto $B$. 
  
  En ese caso escribiremos $f:A \to B$

%%%%%%%%%%%%%%%%%%%%%
{}
\begin{enumerate}[(i)]
  %NUEVO ITEM
  \item 
  Para indicar dicha correspondencia escribimos $y=f(x)$ y decimos que $y$ es el \emph{valor} de $f$ en $x$. 
  \item Al conjunto $A$ se le conoce como dominio. 
  \item Mientras que al conjunto $B$, se le conoce como contradominio.
\end{enumerate}

%%%%%%%%%%%%%%%%%%%%%
{}
  \begin{problema}
   Evaluar $f(x)=x^{2}-3x+2$ en $x=2$
  \end{problema}

  \begin{proof}[Solución]
     \begin{align*} 
	f(2) &= (2)^{2}-3(2)+2 \\
   &=  0
   \end{align*}
  \end{proof}

%%%%%%%%%%%%%%%%%%%%%
{}
  \begin{definicion}
   La \emph{gráfica} de una función $f: A \to B$ es el conjunto 
       \begin{align*}    
   \Gamma_{f} = \sett{\left( a,b \right)\in A\times B}{b=f(a)}
    \end{align*}
  \end{definicion}


%%%%%%%%%%%%%%%%%%%%%
{}
\begin{enumerate}[(i)]
  %NUEVO ITEM
  \item 
  En el caso $A=B=\R^{n}$, diremos que la gráfica de $F:\R \to \R$ es una \emph{curva}. 
  
  \item Diremos que $x\in A$ es la variable \emph{independiente},  mientras que $y\in B$ es la variable \emph{dependiente}.
\end{enumerate}

%%%%%%%%%%%%%%%%%%%%%
\subsection{Polinomios}
 
     \begin{align*}
   f(x) = a_{n}x^{n}+a_{n-1}x^{n-1}+...+a_{0}
   \end{align*}
    
   Si $a_{n}\neq 0$, diremos que $n$ es el grado del polinomio y $a_{n}$, su coeficiente líder. 
   
   
   Denotaremos el grado del polinomio $f$ por $\mathrm{grd}(f)$.

%%%%%%%%%%%%%%%%%%%%%
{}
Si $\mathrm{grd}(f)=n$, la ecuación polinomial $f(x)$ tiene exactamente $n$ raíces (posiblemente repetidas).


\begin{problema}
 Como $x^{3}-3x^{2}+3x-1=0$, se puede reescribir como $(x-1)^{3}$,  entonces la ecuación tiene una raíz $x=1$ repetida 3 veces. 
\end{problema}


%%%%%%%%%%%%%%%%%%%%%
{Teorema del Binomio}
  
     \begin{align*}
   \left( a+x \right)^{n} = 
   a^{n}+\binom{n}{1}a^{n-1}x+\binom{n}{2}a^{n-2}x^{2}+...+x^{n}
   \end{align*}

donde $\binom{n}{k}=\dfrac{n!}{k!\left( n-k \right)!}$

%%%%%%%%%%%%%%%%%%%%%
\subsection{Exponenciales y logaritmos}
{Funciones exponenciales}
     \begin{align*}
   \exp_{a}(x) = a^{x}, a> 0
   \end{align*}
   

%%%%%%%%%%%%%%%%%%%%%
{Leyes de los exponentes}

        \begin{align*}
    \exp_a(m+n) &= 
     \exp_{a}(m)\cdot \exp_{a}(n) \\
     \exp_a(m-n) &=    
     \dfrac{\exp_a(m)}{\exp_a(n)}     \\
     \exp_a(nm) &=      
     \left( \exp_a(m)\right)^{n}  \\       
     \exp_a(0) &=  1 \\     
     \exp_a(1) &=  a     
     \end{align*}

%%%%%%%%%%%%%%%%%%%%%
{Logaritmos}
  La función logarítmica $f(x)=\log_{a}(x)$ es la función inversa de $g(x) = \exp_{a}(x)$, 
   es decir 
       \begin{align*}
    \forall x \in \R: \log_{a}\left( \exp_{a}(x) \right) = x
    \end{align*}
    
         \begin{align*}
     \forall x \in \R^{+}: \exp_{a}\left( \log_{a}(x) \right) = x
     \end{align*}

%%%%%%%%%%%%%%%%%%%%%
{Leyes de los logaritmos}
     \begin{align*}
   \log_{a}(mn)&= \log_{a}(m)\cdot \log_{a}(n) \\
   \log_{a}\left( \dfrac{m}{n} \right)&= \log_{a}(m)-\log_{a}(n) \\ 
   \log_{a}(m^{p}) &= p\log_{a}(m) \\ 
   \log_{a}(1)&= 0\\ 
   \log_{a}(a)&=1
   \end{align*}

%%%%%%%%%%%%%%%%%%%%%
{Logaritmo natural}
  En el caso de que la base sea la constante de Euler, es decir $a=e\approx 2.718$, entonces reescribimos
     \begin{align*}
   \exp_{e}(x)=\exp(x)\\ 
   \log_{e}(x)=\ln(x)
   \end{align*}
   
   Esta última función se conoce como \emph{logaritmo natural}.

%%%%%%%%%%%%%%%%%%%%%
\subsection{Funciones trigonométricas}
{Relaciones fundamentales}
\begin{align*}
\sin(x) &=\cos\left( \dfrac{\pi}{2}-x \right) \\    
\cos(x) &= \sin\left( \dfrac{\pi}{2}-x \right) \\    
\tan(x) &= \dfrac{\sin(x)}{\cos(x)} \\     
\cot(x) &= \dfrac{\cos(x)}{\sin(x)}=\dfrac{1}{\tan(x)} \\    
\sec(x) &= \dfrac{1}{\cos(x)} \\    
\csc(x) &= \dfrac{1}{\sin(x)}
\end{align*}

%%%%%%%%%%%%%%%%%%%%%
\subsection{Identidades pitagóricas}
\begin{align*}
\sin^{2}(x)+\cos^{2}(x)&= 1\\ 
\sec^{2}(x)-\tan^{2}(x)&= 1\\ 
\csc^{2}(x)-\cot^{2}(x)&= 1
\end{align*}

%%%%%%%%%%%%%%%%%%%%%
{Paridad}
     \begin{align*}
   \sin(-x) &= -\sin(x) \\ 
   \cos(-x) &= \cos(x) \\ 
   \tan(-x) &= -\tan(x) 
   \end{align*}

%%%%%%%%%%%%%%%%%%%%%
{Sumas de ángulos}
     \begin{align*}
   \sin(x\pm y) &= \sin(x)\cos(y)\pm\cos(x)\sin(y) \\   
   \cos(x\pm y) &= \cos(x)\cos(y)\mp\sin(x)\sin(y) \\   
   \tan(x\pm y) &= \dfrac{\tan(x)\pm \tan(y)}{1+\mp \tan(x)\tan(y)}
   \end{align*}

%%%%%%%%%%%%%%%%%%%%%
{Ondas sinusoidales}
     \begin{align*}
   \begin{cases}
A\cos(x)+B\sin(x) = \sqrt{A^{2}+B^{2}}\sin(x+\alpha)\\
\tan(\alpha) = \dfrac{A}{B}
\end{cases}
   \end{align*}

%%%%%%%%%%%%%%%%%%%%%
{Periodicidad}
  Las funciones $\sin(x)$ y $\cos(x)$ tiene periodo $T=2\pi$.   
  Mientras que la función $\tan(x)$ tiene periodo $T=\pi$.
  
  
%%%%%%%%%%%%%%%%%%%%%
{Inversas trigonométricas}
   Las funciones inversas de funciones trigonométricas estás sólo definidas \emph{localmente}:  Por ejemplo,
      \begin{align*}
     y = \sin(x) \iff x =\sin^{-1}(y)
     \end{align*} 
     siempre y cuando 
          \begin{align*}
      x\in\left[-\dfrac{\pi}{2}, \frac{\pi}{2} \right],
      y\in\left[ -1,1 \right]
      \end{align*}

%%%%%%%%%%%%%%%%%%%%%
\subsection{Funciones hiperbólicas}
{Relaciones fundamentales}
 \begin{align*}
\sinh(x) &= \dfrac{e^{x}-e^{-x}}{2}\\ 
\cosh(x) &= \dfrac{e^{x}+e^{-x}}{2}
\end{align*}

%%%%%%%%%%%%%%%%%%%%%
{}
\begin{align*}
\tanh(x) &= \dfrac{\sinh(x)}{\cosh(x)}\\ 
&=\dfrac{e^{x}-e^{-x}}{e^{x}+e^{-x}}
 \end{align*}  

%%%%%%%%%%%%%%%%%%%%%
{}
\begin{align*}
\coth(x) &= \dfrac{\cosh(x)}{\sinh(x)}\\ 
& = \dfrac{1}{\tanh(x)}\\ 
& = \dfrac{e^{x}+e^{-x}}{e^{x}-e^{-x}} 
\end{align*}

%%%%%%%%%%%%%%%%%%%%%
{}
 \begin{align*}
\sech(x)&= \dfrac{1}{\cosh(x)} \\ 
&= \dfrac{2}{e^{x}+e^{-x}}
\end{align*}

%%%%%%%%%%%%%%%%%%%%%
{}
\begin{align*}
\csch(x)&= \dfrac{1}{\sinh(x)} \\ 
&= \dfrac{2}{e^{x}-e^{-x}}
\end{align*}

%%%%%%%%%%%%%%%%%%%%%
{Identidades pitagóricas}
\begin{align*}
\cosh^{2}(x)-\sinh^{2}(x)&= 1\\
\sech^{2}(x)+\tanh^{2}(x)&= 1\\
\coth^{2}(x)-\csch^{2}(x)&= 1
\end{align*}

%%%%%%%%%%%%%%%%%%%%%
{Sumas de ángulos}
 \begin{align*}
\sinh(x\pm y) &= \sinh(x)\cosh(y)\pm \cosh(x)\sinh(y)\\
\cosh(x\pm y) &= \cosh(x)\cosh(y)\pm \sinh(x)\sinh(y)\\
\tanh(x\pm y) &= \dfrac{\tanh(x)\pm \tanh(y)}{1\pm \tanh(x)\tanh(y)}
\end{align*}

%%%%%%%%%%%%%%%%%%%%%
\subsection{Ejemplos}
{}
\begin{problema}
\label{solved 1.4}
Si $f(x)=2x^{2}-3x+5$, encontrar
\begin{enumerate}[(i)]
 %NUEVO ITEM     
 \item $f(h)-f(0)$
 
 \item $f(h-1)-f(-1)$
 
 \item $f(x+h)$
 
 \item $f(x+h)-f(x)$
 
 \item $\dfrac{f(x+h)-f(x)}{h}$
\end{enumerate}
\end{problema}


%%%%%%%%%%%%%%%%%%%%%
{}
  \begin{problema}
   \label{solved 1.5}
   Usando las leyes de los exponentes, demostrar las leyes de los logaritmos.
  \end{problema}


%%%%%%%%%%%%%%%%%%%%%
{}
  \begin{problema}
   \label{solved 1.6}
   Demostrar que 
   \begin{enumerate}[(i)]
     %NUEVO ITEM
     \item $\sin^{2}(x)=\dfrac{1}{2}\left( 1-\cos(2x) \right)$ 
     
     \item $\cos^{2}(x)=\dfrac{1}{2}\left( 1+\cos(2x) \right)$
\end{enumerate}
  \end{problema}


%%%%%%%%%%%%%%%%%%%%%
{}
  \begin{problema}
   \label{solved 1.7}
   Demostrar que \[A \cos(x) + B\sin(x) = \sqrt{A^{2}+B^{2}}\sin(x+\alpha)\] donde $\tan(\alpha)=\dfrac{A}{B}$.
  \end{problema}


%%%%%%%%%%%%%%%%%%%%%
{}
  \begin{problema}
   \label{solved 1.8}
   Demostrar que 
   \begin{enumerate}[(i)]
     %NUEVO ITEM
     \item $\cosh^{2}(x)-\sinh^{2}(x)=1$       
     \item $\sech^{2}(x)+\tanh^{2}(x)=1$
\end{enumerate}
  \end{problema}


%%%%%%%%%%%%%%%%%%%%%
{}
  \begin{problema}
   \label{solved 1.9}
   Demostrar que $\cosh^{-1}(x)=\pm \ln\left( x+\sqrt{x^{2}-1} \right)$
  \end{problema}


%%%%%%%%%%%%%%%%%%%%%
