\section{Análisis combinatorio}

El \emph{análisis combinatorio} es una manera sofisticada de contar.

\begin{proposicion}[Principio fundamental del conteo y diagramas de árbol]
	Si una tarea se puede realizar en $n$ formas diferentes y otra en $m$ formas diferentes, entonces las dos tareas se pueden realizar en $n\times m$ formas diferentes.	
\end{proposicion}

\begin{ejemplo}
	\label{exmp:1.14}
\end{ejemplo}
\begin{enumerate}
	\item Si una persona tiene 2 camisas y 4 corbatas, ?`de cuantas formas puede combinarlas?
	\item Construya un diagrama de árbol para representar todas estas opciones.
\end{enumerate}

\subsection{Permutaciones}

Consideremos un conjunto finito $X=\set{x_{1},...,x_{N}},$ esto es, $X$ tiene \emph{cardinalidad} $n(X)=N < \infty.$

Una función biyectiva (invertible) $\sigma:X \to X$ es llamada \emph{permutación} en $X.$

Observe que las composiciones e inversas de permutaciones, así como la identidad, son tambi\'en permutaciones. En este caso, diremos que la permutación $\sigma$ \emph{actua} en $X.$

Supongamos que la permutación $\sigma$ actúa en $X={x_{1},x_{2},x_{3}}$ de la siguiente manera:
$$
\sigma(x_{1})=x_{2}, \; \sigma(x_{2})=x_{3}, \; \sigma(x_{3})=x_{1}.
$$

Entonces, podemos representar la permutación de la siguiente manera
$$\sigma=
\begin{pmatrix}
	1 & 2 & 3 \\
	2 & 3 & 1,
\end{pmatrix},
$$  es decir, sólo nos fijamos de que manera actúa en el índice $j$ del elemento $x_{j}.$



De manera general, numerando los elementos de $X=\set{x_{1},...,x_{N}},$ podemos identificar este conjunto con $A_{N}=\set{1,...,N}$ por medio de la biyección $x_{i} \mapsto i.$


Ahora, consideremos una permutación $\sigma:A_{N}\to A_{N},$ tal que $\sigma(i)=\sigma_{i}.$ Entonces podemos representa $\sigma$ por medio de 
$$\sigma=
\begin{pmatrix}
	1& ... & N \\
	\sigma_{1}& ... & \sigma_{N}.
\end{pmatrix}
$$


El conjunto de todas las permutaciones $:A_{N}\to A_{N}$ se denota por $S_{N}$ y tiene una cardinalidad 
$n(S_{N})=N!.$

\subsection{Cálculos con permutaciones}
Si tenemos $n$ objetos distintos y queremos ordenarlos tendremos
\begin{align*}
	n \times (n-1) \times ... 2\times 1
\end{align*} formas diferentes de hacerlo.

{}
\begin{definicion}[$n$ factorial]
	\[
		n! = \begin{cases}
			1 & n=0 \\
			n\times(n-1)! & n>0
		\end{cases}
	\]
	
\end{definicion}


Si tenemos $n$ objetos distintos y queremos arreglar $r$ de estos en una linea, entonces tendremos una \emph{permutación} de $n$ en $r$ dada por
\begin{align}
	\label{1.25}
	P^{n}_{r}=n\times(n-1)\times...\left( n-r+1 \right)
\end{align} 
o de manera equivalente
\begin{align}
	\label{1.27}
	P^{n}_{r}=\dfrac{n!}{(n-r)!}
\end{align}



\subsection{Combinaciones}
{}
En una \emph{permutación}, uno está interesado en el orden de los objetos. Así $abc$ y $bca$ son permutaciones diferentes.  Pero en algunos problemas, uno está interesado sólo en elegir objetos sin importar su orden.  Tales selecciones se llaman \emph{combinaciones}.  Por ejemplo, $abc$ y $bca$ representan la misma combinación.



El número de combinación $ C^{n}_{r} $ al elegir $r$ objetos de una colección de $n$ diferentes está dada por el \emph{número combinatorio}
\begin{align}
	\label{1.29}
	C^{n}_{r} = \comb{n}{r}=\dfrac{n!}{r!\left( n-r \right)!}
\end{align}



{Algunas fórmulas combinatorias}
\begin{align}
	\label{1.30}
	\comb{n}{r}&=\dfrac{P(n,r)}{r!} \\
	\label{1.31}
	\comb{n}{r}&=\comb{n}{n-r}\\
	\comb{n}{r}&=\comb{n-1}{r-1}+\comb{n-1}{r}
\end{align}


\subsection{El Teorema del Binomio}

\begin{teorema}[Teorema del binomio]
	\begin{align}
		\label{1.32}
		\left( x+y \right)^{n} = \sum_{r=0}^{n}\comb{n}{r}x^{r}y^{n-r}
	\end{align}
	
\end{teorema}

%\begin{proposicion}[Aproximación de Stirling]
%	\begin{align*}
%		\label{1.33}
%		n! \approx \sqrt{2\pi n}\left( n^{n}e^{-n} \right)
%	\end{align*}
%\end{proposicion}




