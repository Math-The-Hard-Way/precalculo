\section{Números complejos}

En estas notas, denotaremos por $\R$ el conjunto de números reales. En esta sección, procederemos de manera informal,
para motivar la definición de un número complejo y formalizar sus propiedades, en secciones posteriores. 


Supongamos que $a,b,c\in \R,$ y queremos resolver
la ecuación
$$
ax^2+bx+c=0.
$$

De manera algebraíca encontramos que las soluciones estan dadas por la fórmula
$$
x=\dfrac{-b\pm \sqrt{D}}{2a}, \, D=b^2-4ac.
$$

Si $D \geq 0,$ entonces $D$ es un número real. Sin embargo, ¿qué sucede si $D<0$?. Por la \emph{ley de los signos} si
$x,y\geq 0,$ entonces $xy\geq 0.$ De la misma manera, si $x,y<0,$ entonces $xy>0.$ En particular, para cualquier
$x\in \R,$ tenemos que $x^{2}=x\cdot x\geq 0.$ Por lo tanto, $\sqrt{D} \notin \R$ si $D<0.$

Una solución a este problema es definir el número $i=\sqrt{-1}.$ En este caso, si $D<0,$ entonces usando leyes de los
exponentes tenemos que
$$
\sqrt{D}=\sqrt{(-1)(-D)}=\sqrt{-1}\sqrt{-D}=\sqrt{-D}i.
$$
En este caso, como $D<0,$ entonces $-D>0$ y $\sqrt{-D}\in \R.$

\begin{problema}
 Las soluciones de la ecuación $x^2+1=0$ son $x=0+i1$ y $x=0+i(-1),$ o simplemente, $x=\pm i.$
\end{problema}

\begin{problema}
 Encuentre las soluciones de la siguientes ecuaciones:
 \begin{enumerate}
  \item $x^{4}+16=0,$
  \item $x^{2}-2x+2=0.$
 \end{enumerate}

\end{problema}


 Entonces, diremos que un número complejo es una cantidad de la forma
 $$
z=x+iy, \, x,y\in \R, \, i=\sqrt{-1}.
 $$
 Observe que si $x\in \R,$ podemos identificarlo con $x+i0.$

Definimos la suma de números complejos $z=x+iy,z'=x+iy'$ de la siguiente manera:
$$
z+z'=(x+x')+i(y+y').
$$

\begin{problema}
Demuestre que 
\begin{enumerate}
 \item $(x+iy)+(x'+iy')=(x'+iy')+(x+iy).$
 \item $\left[ (x+iy)+(x'+iy') \right] +(x''+iy'')= (x+iy)+\left[ (x'+iy') +(x''+iy'') \right]$
 \item $0+(x+iy)=x+iy$
 \item $(x+iy)+((-x)+i(-y))=0$
\end{enumerate}\end{problema}

Diremos que $0=0+i0$ es el \emph{neutro aditivo} en los número complejos y que $-z:=-x-iy$ es el \emph{inverso aditivo}
de $z=x+iy.$

Ahora queremos definir la multiplicación $(x+iy)(x'+iy').$ Sigamos las reglas algebraicas usuales para números reales,
salvo por la identidad $i^2=-1.$

\begin{align*}
(x+iy)(x'+iy')&= x(x'+iy')+iy(x'+iy')\\
&= xx'+x(iy')+(iy)x'+(iy)(iy') \\
&= xx' + ixy +iyx' + i^{2}yy' \\
&= (xx'-yy')+i(xy'+yx').
\end{align*}

En resumen,
 $$
zz'= (xx'-yy')+i(xy'+yx') \in \C.
 $$

\begin{problema}
Demostrar las siguientes propiedades de la multiplicación de número complejos
\begin{enumerate}
 \item $(x+iy)(x'+iy')=(x'+iy')(x+iy).$
 \item $\left[ (x+iy)(x'+iy') \right] (x''+iy'')= (x+iy)\left[ (x'+iy') (x''+iy'') \right]$
 \item $(1+i0)(x+iy)=x+iy$
 \item $(x+iy)(x-iy)=x^2+y^2.$
 \item $(x+iy)\left( \dfrac{x-iy}{x^2+y^2} \right)=1.$
\end{enumerate}
\end{problema}

Diremos que $1=1+i0$ es el \emph{neutro multiplicativo} en los número complejos y que $$
z^{-1}:=\dfrac{x-iy}{x^2+y^2} 
$$ es el \emph{inverso multiplicativo} de $z=x+iy.$

Si definimos $\bar{z}=x-iy,$ para $z=x+iy,$ podemos reescribir $$z^{-1}=\dfrac{\bar{z}}{z\bar{z}}.$$
Diremos que $\bar{z}$ es el \emph{conjugado} de $z.$

\begin{observacion}
Los número reales se pueden identificar con una línea recta. Como $i$ no se puede identificar con un número en la línea
recta, se decía que este número era \emph{imaginario.} Sin embargo, podemos visualizar los números complejos (es decir,
¡dibujarlos!), para lo cuál necesitaremos ``más espacio''. Como requerimos dos números reales para describir un
complejo, tendremos que dibujarlos en el plano.
\end{observacion}

\begin{problema}
\label{exe:1.1.1}
 Encuentre el resultado de las siguientes operaciones:
 \begin{enumerate}
  \item $\left( 1+i\sqrt{3} \right)\left( -1 +i\sqrt{3} \right)$
  \item $\dfrac{\frac{1}{\sqrt{2}}+i\frac{1}{\sqrt{2}}}{\sqrt{3}+i1}$
  \item $\left( \sqrt{2}+i\sqrt{6} \right)^{3}$
 \end{enumerate}

\end{problema}

\subsection{Estructura algebraica de $\C$}

\begin{definicion}El \emph{plano} es el conjunto
	$$
	\R^{2}=\set{(x,y)|x,y\in \R },
	$$
	de parejas ordenadas de números reales.
\end{definicion}

En este espacio, podemos definir varias operaciones. Cuando al conjunto lo dotamos de ciertas operaciones, decimos que
es un \emph{estructura (matemática)} en el plano. Una de las más importantes es la estructura de \emph{espacio
	vectorial,} que a continuación presentamos.

\begin{definicion} El \emph{plano euclideano} es $\R^{2}$ dotado de las siguiente operaciones:
	\begin{enumerate}
		\item 
		$
		(x,y)+(x',y')=(x+x',y+y'),
		$
		\item Si $\a \in \R,$
		$
		\a\cdot(x,y)=(\a x,\a y),
		$
	\end{enumerate}
\end{definicion}

\begin{observacion}
	En este caso, a los pares ordenados $(x,y)\in \R^{2}$ les llamaremos \emph{vectores (en el plano)}, mientras que a los
	números reales los llamaremos \emph{escalares.} Entonces, nos referiremos a la primera operación como \emph{suma de
		vectores,} mientras que a la segunda como \emph{multiplicación por escalares.} Estas son las operaciones \emph{usuales}
	en el plano euclideano.
\end{observacion}


\begin{problema}
	Encuentre y grafique los vectores resultantes.
	\begin{enumerate}
		\item $2(1,0)+3(0,1),$
		\item $\frac{1}{5}(5,0)-1(0,2).$
	\end{enumerate}
	
\end{problema}

Con el plano euclideano en mente, podemos definir de manera formal el conjunto de número complejos. Observe que
podríamos identificar $x+iy$ con el vector $(x,y).$ Observe que con esta identificación, el resultado de la suma de
números complejos coincide con la de vectores. De igual manera, podemos identificar la multiplicación entre número
complejo. Esto nos lleva a la definición formal de \emph{números complejos.}

\begin{definicion}
	El \emph{campo} de número complejos $\C$ es el conjunto $\R^{2}$ dotado de las siguientes operaciones:
	\begin{enumerate}
		\item
		$
		(x,y)+(x',y')=(x+x',y+y'),
		$
		\item
		$
		(x,y)(x',y')=(xx'-yy',xy'+yx').
		$
	\end{enumerate}
\end{definicion}


Si identificamos $\a \in \R,$ con $(a,0) \in \C,$ resulta que la multiplicación por escalares coindice con la
multiplicación de números complejos para escalares reales, es decir, si $a\in \R,$
$$
a(x,y)=(a,0)(x,y).
$$

\begin{problema}
	Verifique la afirmación anterior.
\end{problema}


\begin{problema}
	Verifque las siguiente propiedades. Si $u,v,w \in \C \cong \R^{2},$ entonces
	\begin{enumerate}
		\item $u+v\in \C$ 
		\item $(u+v)+w=u+(v+w)$
		\item $u+v=v+u$
		\item Existe $0\in \C,$ tal que $u+0=0$
		\item Para cada $u\in \C,$ existe $-u\in\C,$ tal que $u+(-u)=0$
		\item $uv \in \C$
		\item $(uv)w=u(vw)$
		\item $uv=vu$
		\item Existe $1\in \C,$ tal que $1u=u$
		\item Para cada $u\in C, u \neq 0,$ existe $u^{-1}\in \C,$ tal que $u u^{-1}=1$
		\item $u(v+w)=uv+uw.$
	\end{enumerate}
	
\end{problema}

\begin{observacion}
	Cualquier conjunto $S,$ con operaciones suma y multiplicación, que cumplan las propiedades anteriores, se conoce como
	un \emph{campo.} Otros ejemplos de campos son las fracciones y los mismos números reales. En teoría número, ejemplos de
	campos son los enteros \emph{módulo} $p$ $\mathbb{Z}_{p},$ con $p$ un número primo.
\end{observacion}

\subsection{Forma polar de los números complejos}

En la presente sección, suponemos que el lector tiene conocimientos elementales de trigonometría y geometría analítica. 

Como los números complejos son vectores, podemos calculas su longitud o \emph{norma.}

\begin{definicion}
	Si $z=x+iy\in \C,$ entonces la norma de $z$ se define como
	$$
	\norm{z}=\sqrt{x^{2}+y^{2}}.
	$$
\end{definicion}

Como hicimos antes, definimos de manera formal el \emph{conjugado} de un número complejo.
\begin{definicion}
	Si $z=(x,y)\in \C,$ su \emph{conjugado} esta dado por
	$$
	\bar{z}=(x,-y)\in \C.
	$$
\end{definicion}

De manera que $\norm{z}^{2}=z\bar{z}.$

De la misma manera, siendo un vector podemos medir el ángulo que abre respecto al vector $(1,0)$, en el sentido de las
manecillas del reloj, al cual llamaremos \emph{argumento} y definimos analíticamente de la siguiente forma.
\begin{definicion}
	El argumento $\theta(z)$ de $z=x+iy\in \C$ se define como
	\begin{enumerate}
		\item $\arctan\left( \dfrac{y}{x} \right)$ si $x>0$
		\item $\pi + \arctan\left( \dfrac{y}{x} \right)$ si $x<0$
		\item $\dfrac{\pi}{2}$ si $x=0, y> 0$
		\item $-\dfrac{\pi}{2}$ si $x=0, y< 0$
	\end{enumerate}
	
\end{definicion}

\begin{definicion}
	Si $z\in \C$ tiene norma $r>0$ y argumento $\th,$ decimos que $$
	z=r\arg{\th},
	$$
	es la \emph{forma polar} de $z,$ donde $\arg{\cdot}:\R\to\R^{2},$
	$$
	\arg{\th}=\, \left( \cos(\th), \sin (\th) \right).
	$$
\end{definicion}

\begin{problema}
	Demostrar que
	\begin{enumerate}
		\item $\arg{0}=1$
		\item $\arg{\th+2\pi}=\arg{\th}$
		\item $\overline{\arg{\th}}=\arg{-\th}$
		\item $\arg{\th+\tau}=\arg{\th}\arg{\tau}$
		\item $\arg{n\th}=\left( \arg{\th} \right)^{n}$
	\end{enumerate}
	
\end{problema}

\begin{problema}
	\begin{enumerate}
		\item Si $z=r\arg{\th} \in \C,$ entonces
		\begin{enumerate}
			\item   $
			z^{-1}=r^{-1} \overline{\arg{\th}}
			$
			\item Si $n$ es un número entero, $
			z^{n}=r^{n}\arg{n \th}.
			$
		\end{enumerate}
		
		
		\item Si $z=r\arg{\th}, w=s\arg{\tau} \in \C,$ entonces
		\begin{enumerate}
			\item   $
			zw=rs\arg{\th+\tau}.
			$
			\item
			$
			\dfrac{z}{w}=\dfrac{r}{s}\arg{\th-\tau}
			$
		\end{enumerate}
		
		
		
		
		
	\end{enumerate}
	
\end{problema}

Esta última identidad se conoce como \emph{identidad de De Moivre.}


\begin{problema}
	Convierta a su forma polar, cada uno de los números en el ejercicio \ref{exe:1.1.1} y realice las operaciones
	correspondientes, usando los resultados anteriores. 
\end{problema}


