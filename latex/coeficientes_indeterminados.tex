\section{M\'etodo de Coeficientes Indeterminados}


Por el teorema \ref{bron:thm:8.4}, la soluci\'on de $L(y)=0$ está dada por la soluci\'on particular $y_{p}$ más la soluci\'on general $y_{h},$ la cuál es la soluci\'on de la ecuaci\'on lineal homog\'enea $L(y)=0.$



En esta secci\'on, aprenderemos a obtener $y_{p},$ una vez que $y_{h}$ es conocida, a trav\'es del \emph{coeficientes indeterminados.}


\subsection{Forma simple del m\'etodo}


\begin{observacion}
	Para aplicar este m\'etodo a la ecuaci\'on diferencial $L(y)=\phi(x),$ \emph{$\phi$ y TODAS sus derivadas} deben estar generadas por un conjunto \emph{finito} de funciones linealmente independientes
	$$
	\set{y_{1}(x),...,y_{n}(x)}.
	$$
\end{observacion}



En ese caso, comenzaremos suponiendo que $y_{p}(x)$ es una combinanci\'on lineal de $y_{1}(x),...,y_{n}(x):$
$$
y_{p}(x)=A_{1}y_{1}(x)+...+A_{n}y_{n}(x)
$$
donde $A_{1},A_{2},...,A_{n}$ son constantes.



A continuaci\'on, revisaremos algunos casos comunes, en los que podemos aplicar dicho m\'etodo.



\subsection{Caso $\phi(x)=p_{n(x)}$}
Si suponemos que $\phi(x)$ es un polinomio de grado $n,$ entonces
\[
	\label{bron:11.1}
	y_{p}(x)=A_{n}x^{n}+A_{n-1}x^{n-1}+...+A_{1}x+A_{0}.
\]




{}
Recordemos que la solución general de la ecuación $y''-y'-2y=0$ está dada por...
\begin{align*}
	y_{h}(x)=c_{1}e^{-x}+c_{2}e^{2x}
\end{align*}

{}
%\begin{figure}
%	\centering
%	\includegraphics[width=10cm,keepaspectratio=true]{../ed_ejemplos/ed_ejemplos_p2/img_2-7.png}
%	img_2-7.png: 0x0 pixel, 300dpi, 0.00x0.00 cm, bb=
%	\label{fig:2-7}
%\end{figure}



\begin{problema}
	\label{bron:exmp:11.1}
	Resuelva
	$$
	y''-y'-2y=4x^{2}
	$$
\end{problema}




\subsection{Caso $\phi(x)=ke^{\a x}$}
Si suponemos que $\phi(x)$ es una funci\'on exponencial, entonces
\[
	\label{bron:11.1}
	y_{p}(x)=Ae^{\a x}.
\]





\begin{problema}
	\label{bron:exmp:11.2}
	Resuelva
	$$
	y''-y'-2y=e^{3x}.
	$$
\end{problema}




\subsection{Caso $\phi(x)=ke^{\a x}$}
Si suponemos que $\phi(x)$ es una funci\'on exponencial, entonces
\[
	\label{bron:11.2}
	y_{p}(x)=Ae^{\a x}.
\]




\subsection{Caso $\phi(x)=k_{1}\sin(\b x)+k_{2}\cos(\b x)$}
Si suponemos que $\phi(x)$ es una funci\'on senoidal, entonces
\[
	\label{bron:11.3}
	y_{p}(x)=A\sin(\b x)+B\cos(\b x)
\]




\begin{problema}
	\label{bron:exmp:11.3}
	Resuelva
	$$
	y''-y'-2y=\sin(2x).
	$$
\end{problema}



\subsection{Generalizaciones}


Si $\phi(x)=e^{\a x}p_{n}(x),$ entonces
\[
	\label{bron:11.4}
	y_{p}=e^{\a x}\left( A_{n}x^{n}+...+A_{1}x +A_{0}\right).
\]



{}
\begin{problema}
	Resolver
	\begin{align*}
		y'' = e^{-x}x^2
	\end{align*}
	
\end{problema}




Si $\phi(x)=ke^{\a x}\sin(\b x)$ o $\phi(x)=ke^{\a x}\cos(\b x),$ entonces
\[
	\label{bron:11.5}
	y_{p}(x)=A_{0}e^{\a x}\sin(\b x)+B_{0}e^{\a x}\cos(\b x).
\]



{}
\begin{problema}
	Resuelva 
	\begin{align*}
		y'' = e^{-x}\cos(3x)
	\end{align*}
	
\end{problema}




Aun más, si $\phi(x)=ke^{\a x}\sin(\b x)p_{n}(x)$ o $\phi(x)=ke^{\a x}\cos(\b x)p_{n}(x),$ entonces
\[
	\begin{split}
		\label{bron:11.5}
		y_{p}(x)=A_{0}e^{\a x}\sin(\b x)\left( A_{n}x^{n}+...+A_{1}x +A_{0}\right)\\+B_{0}e^{\a x}\cos(\b x)\left( B_{n}x^{n}+...+B_{1}x +B_{0}\right).
	\end{split}
\]




\begin{observacion}
	Si cualquier t\'ermino de $y_{p},$ salvo por los t\'erminos constantes, es tambi\'en un t\'ermino de $y_{h},$ entonces $y_{p}$ \emph{debe ser modificada} multiplicandola por $x^{m}.$
	
	
	Aqu\'i $m$ es el entero positivo más pequeño tal que el producto $x^{m}y_{p}$ no tiene t\'erminos en común con $y_{h}.$
\end{observacion}



%%%%%%%%%%%%%%%%%
{}
\begin{problema}
	Resuelve la ecuación
	\begin{align*}
		y''-y'-2y = \dfrac{1}{2}e^{-x}
	\end{align*}
	
\end{problema}


%%%%%%%%%%%%%%%%%% 

\begin{observacion}
	Si  $\phi(x)$ no tiene alguna de las formas anteriores o la ecuaci\'on diferencial no tiene coeficientes constantes, este m\'etodo no se puede aplicar.
\end{observacion}


\subsection{Principio de superposición}

%%%%%%%%%%%%%%%%%
{}
Consideremos la ecuación diferencial 
\begin{align*}
	\label{superposicion}
	L[y] = \phi_{1}(x) + \phi_{2}(x)
\end{align*}
donde $L$ es un operador diferencial lineal con coeficientes constantes de la forma 
\begin{align*}
	L[y] = y^{(n)}+...+a_{1}y'+a_{0}y
\end{align*}



%%%%%%%%%%%%%%%%%
{}
Digamos que $y_{h}(x)$ es la solución de la ecuación homogénea asociada, es decir, $$L[y_{h}]=0,$$ mientras que $y_{p1(x)}$ resuelve $$L[y_{p1}]=\phi_{1}(x)$$ y $y_{p2}(x)$, $$L[y_{p2}]=\phi_{2}(x).$$

%%%%%%%%%%%%%%%%%
{}
Entonces $y(x)=y_{h}(x)+y_{p1}(x)+y_{p2}(x)$ resuelve la ecuación 
$$
L[y]=\phi_{1}(x)+\phi_{2}(x).
$$

%%%%%%%%%%%%%%%%%
{}
\begin{problema}
	Resuelve la ecuación
	\begin{align*}
		y''-y'-2y=2e^{3x}-3t^{2}
	\end{align*}
	
\end{problema}


%%%%%%%%%%%%%%%%%% 
