\section*{Problemas}

%\subsection*{Proposiones y Tablas de Verdad}


\begin{problema}
	Sea $p:\texttt{``Hace frío''}$ y $q:\texttt{``Está lloviendo''.}$ Proponga un enunciado verbal simple que describa cada una de las siguientes proposiciones:
	\begin{enumerate}
		\item $\neg p;$
		\item $p \wed q;$
		\item $p \vee q;$
		\item $q \vee \neg p.$
	\end{enumerate}
	
\end{problema}




\begin{problema}
	Encuentre la tabla de verdad de $\neg p \wed q.$
\end{problema}




\begin{problema}
	Demuestre que la propisición 
	$$
	p \vee \neg \left( p\wed q \right)
	$$ es una tautología.
\end{problema}




\begin{problema}
	Muestre que las proposiciones $\neg\left( p \wed q \right)$ y $\neg p \vee \neg q$ son lógicamente equivalentes.
\end{problema}




\begin{problema}
	Use las leyes en la tabla \ref{fig:tabla:4.1} para mostrar que 
	$$
	\neg \left( p \wed q \right) \vee \left( \neg p \wed  q \right) \equiv \neg p
	$$
\end{problema}



%\subsection*{Sentencias condicionales}


\begin{problema}
	\label{lip:sol:4.6}
	Reescriba los siguientes enunciados sin usar el condicional:
	\begin{enumerate}
		\item Si hace frío, el usa sombrero. 
		\item Si la productividad se incrementa, entonces el salario aumenta.
	\end{enumerate}
	
\end{problema}




\begin{problema}
	\label{lip:sol:4.7}
	Considere la proposición condicional $p \imply q.$ La proposiciones 
	\begin{center}
		${\color{red}q \imply p,} {\color{blue}\, \neg p \imply \neg q,} \, {\color{green}\neg q \imply \neg p}$
	\end{center}
	son llamadas {\color{red} conversa,} {\color{blue}inversa} y {\color{green} contrapositiva}, respectivamente.
	
	
	?`Cuáles de estas proposiciones son lógicamente equivalente s a $p\imply q$?
\end{problema}




\begin{problema}
	Determine la contrapositiva de cada enunciado:
	\begin{enumerate}
		\item Si Erik es poeta, entonces es pobre. 
		\item Solo si Marcos estudia, pasará el examen. 
	\end{enumerate}
	
\end{problema}




\begin{problema}
	Escriba la negación de cada enunciado, tan simple como sea posible:
	\begin{enumerate}
		\item Si ella trabaja, ganará dinero. 
		\item El nada si y solo si el agua está tibia. 
		\item Si neva, entonce no manejar\'e.
	\end{enumerate}
	
\end{problema}



%\subsection*{Argumentos}


\begin{problema}
	Muestre que el siguiente argumento es una falacia:
	$$
	p\imply q, \neg p \yields \neg q.
	$$
\end{problema}




\begin{problema}
	Muestre que el siguiente argumento es válido:
	$$
	p\imply q, \neg q \yields \neg p.
	$$
\end{problema}




\begin{problema}
	Muestre que el siguiente argumento siempre es válido:
	$$
	p \imply \neg q, r \imply q, r \yields \neg p.
	$$
\end{problema}




\begin{problema}
	Determine la validez del siguiente argumento:
	\begin{center}
		\begin{tabular}{l}
			Si $7$ es menor que $4$, entonces $7$ no es número primo\\
			$7$ no es menor que $4$\\\hline
			$7$ no es número primo.
		\end{tabular}
	\end{center}
	
\end{problema}




\begin{problema}
	Determine la validez del siguiente argumento:
	\begin{center}
		\begin{tabular}{l}
			Si dos lados de un triángulo son iguales, entonces los respectivos ángulos opuestos son iguales\\
			Dos lados de un triángulo no son iguales\\\hline
			Los respectivos ángulos opuestos no son iguales.
		\end{tabular}
	\end{center}
	
\end{problema}



%\subsection*{Cuantificadores y Funciones Proposicionales}


\begin{problema}
	Sea $A=\set{1,2,3,4,5}.$ Determine el valor de verdad de cada uno de los siguientes enunciados:
	\begin{enumerate}
		\item $\exists x \in A: x+3=10;$ 
		\item $\forall x \in A: x+3<10;$ 
		\item $\exists x \in A: x+3<5;$ 
		\item $\forall x \in A: x+3 \leq 7.$
	\end{enumerate}
	
\end{problema}




\begin{problema}
	Determine el valor de verdad de cada uno de las siguientes afirmaciones donde $U=\set{1,2,3}$ es el conjunto \emph{``universo''} (de referencia):
	\begin{enumerate}
		\item $\exists x \forall y: x^{2}< y+1;$ 
		\item $\forall x \exists y: x^{2}+y^{2}<12;$ 
		\item $\forall x \forall y: x^{2}+y^{3}<12.$
	\end{enumerate}
	
\end{problema}




\begin{problema}
	Encuentre la negación de cada una de las siguientes afirmaciones:
	\begin{enumerate}
		\item $\exists x \forall y: p(x,y);$ 
		\item $\forall x \forall y: p(x,y);$ 
		\item $\exists x \exists y \forall z: p(x,y,z).$
	\end{enumerate}
	
\end{problema}




\begin{problema}
	Sea $$p(x): x+2>5.$$ Indique cuando $p(x)$ es una función proposicional o no en cada uno de los siguientes conjuntos: 
	\begin{enumerate}
		\item $\N$ 
		\item $\Z^{-}=\set{-1,-2,-3,...}$ 
		\item $\mathbb{C}$
	\end{enumerate}
	
\end{problema}




\begin{problema}
	Niegue cada uno de las siguientes afirmaciones:
	\begin{enumerate}
		\item Todos los estudiantes viven en los dormitorios.
		\item A todos los estudiantes de ingeniería le gusta el futbol.
		\item Algunos estudiantes tienen 25 años o más.
	\end{enumerate}
\end{problema}
