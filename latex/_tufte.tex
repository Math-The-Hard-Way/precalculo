% chapter format

\usepackage{amsthm,thmtools,xcolor}
\setcounter{tocdepth}{1}
\setcounter{secnumdepth}{1}
%% paleta 1
%\definecolor{mycolor1}{RGB}{8,32,50}
%\definecolor{mycolor2}{RGB}{44,57,75}
%\definecolor{mycolor3}{RGB}{51,71,86}
%\definecolor{mycolor4}{RGB}{255,76,41}

%%% paleta 2
%\definecolor{mycolor1}{RGB}{17,29,94}
%\definecolor{mycolor2}{RGB}{199,0,57}
%\definecolor{mycolor3}{RGB}{243,113,33}
%\definecolor{mycolor4}{RGB}{192,226,24}

%% paleta 3
%\definecolor{mycolor1}{RGB}{75,93,103}
%\definecolor{mycolor2}{RGB}{50,47,61}
%\definecolor{mycolor3}{RGB}{89,64,92}
%%\definecolor{mycolor4}{RGB}{135,85,111}
%
%% paleta 4
%\definecolor{mycolor1}{RGB}{26,26,46}
%\definecolor{mycolor2}{RGB}{22,33,62}
%\definecolor{mycolor3}{RGB}{15,52,96}
%\definecolor{mycolor4}{RGB}{233,69,96}

% paleta 4
\definecolor{mycolor1}{RGB}{255, 201, 150}
\definecolor{mycolor2}{RGB}{255, 132, 116}
\definecolor{mycolor3}{RGB}{159, 95, 128}
\definecolor{mycolor4}{RGB}{88, 61, 114}

\colorlet{ColorVariable1}{mycolor1}
\colorlet{ColorVariable2}{mycolor2}
\colorlet{ColorVariable3}{mycolor3}
\colorlet{ColorVariable4}{mycolor4}

\hypersetup{
	pdftitle={Precálculo},
	pdfauthor={Juliho Castillo Colmenares},
	colorlinks=true,	
		linkcolor=ColorVariable4,
		anchorcolor=ColorVariable4,
		runcolor=ColorVariable4,		
		filecolor=ColorVariable4,
		citecolor = ColorVariable4,      
		urlcolor=ColorVariable4,
	frenchlinks=true
}

\titleformat{\chapter}%
{\huge\rmfamily\itshape\color{ColorVariable4}}% format applied to label+text
{\llap{\colorbox{ColorVariable4}{\parbox{1.5cm}{\hfill\itshape\huge\color{white}\thechapter}}}}% label
{2pt}% horizontal separation between label and title body
{}% before the title body
[]% after the title body

% section format
\titleformat{\section}%
{\normalfont\Large\itshape\color{ColorVariable4}}% format applied to label+text
{\llap{\colorbox{ColorVariable4}{\parbox{1.5cm}{\hfill\color{white}\thesection}}}}% label
{1em}% horizontal separation between label and title body
{}% before the title body
[]% after the title body

% subsection format
\titleformat{\subsection}%
{\normalfont\large\itshape\color{ColorVariable4}}% format applied to label+text
{\llap{\colorbox{ColorVariable4}{\parbox{1.5cm}{\hfill\color{white}\thesubsection}}}}% label
{1em}% horizontal separation between label and title body
{}% before the title body
[]% after the title body


%\newenvironment{problema}{
%	\medskip
%	%\begin{framed}
%	\bgroup\color{ColorVariable1}
%	{\textbf{Problema.}}
%}{
%	\egroup
%	%\end{framed}
%	\medskip
%}

\declaretheoremstyle[
headfont=\color{ColorVariable1}\normalfont\bfseries,
bodyfont=\color{ColorVariable2}\normalfont\itshape,
]{colored}

\declaretheoremstyle[
headfont=\color{ColorVariable3}\normalfont\bfseries,
bodyfont=\color{ColorVariable3}\normalfont\itshape,
]{colored-2}

\declaretheorem[
style=colored,
name=Problema,
numberwithin=section, 
%shaded={rulecolor=Lavender,
%	rulewidth=2pt, 
%	bgcolor={rgb}{1,1,1}}
]{problema}

\declaretheorem[
style=colored,
name=Solución,
numbered = no
]{solucion}

\declaretheorem[
style=colored,
name= Ejemplo,
numberwithin=section
]{ejemplo}

\declaretheorem[
style=colored-2,
name= Definición,
numbered = no
]{definicion}

\newenvironment{algoritmo}[1]{
	\medskip
	%\begin{framed}
	\bgroup
	\color{ColorVariable3}
	{\textbf{Algoritmo. (#1)}}
	\ttfamily
}{
	\egroup
	%\end{framed}
	\medskip
}

%\newtheorem{teorema}{Teorema}%[chapter]

\declaretheorem[
style=colored,
name=Teorema,
]{teorema}

%\newtheorem{proposicion}{Proposición}%[chapter]

\declaretheorem[
style=colored,
name=Proposición,
]{proposicion}

%\newtheorem{observacion}{Observación}%[chapter]

\declaretheorem[
style=colored-2,
name=Observación,
numbered = no
]{observacion}

\newtheorem{axioma}{Axioma}%[chapter]
\newtheorem{sugerencia}{Sugerencia}%[chapter]
\newtheorem{corolario}{Corolario}%[chapter]