
\usepackage{amsthm,thmtools,xcolor}
\setcounter{tocdepth}{3}
\setcounter{secnumdepth}{3}

\definecolor{mycolor1}{RGB}{251, 54, 64}
\definecolor{mycolor2}{RGB}{96, 95, 94}
\definecolor{mycolor3}{RGB}{29, 52, 97}
\definecolor{mycolor4}{RGB}{31, 72, 126}
\definecolor{mycolor5}{RGB}{36, 123, 160}

\colorlet{ColorVariable1}{mycolor1}
\colorlet{ColorVariable2}{mycolor2}
\colorlet{ColorVariable3}{mycolor3}
\colorlet{ColorVariable4}{mycolor4}
\colorlet{ColorVariable5}{mycolor5}

\hypersetup{
	pdftitle={Precálculo},
	pdfauthor={Juliho Castillo Colmenares},
	colorlinks=true,
	linkcolor=ColorVariable5,
	anchorcolor=ColorVariable5,
	runcolor=ColorVariable5,
	filecolor=ColorVariable5,
	citecolor = ColorVariable5,
	urlcolor=ColorVariable5,
	frenchlinks=true
}

\titleformat{\chapter}%
{\huge\rmfamily\itshape\color{ColorVariable1}}% format applied to label+text
{\llap{\colorbox{ColorVariable1}{\parbox{1.5cm}{\hfill\itshape\huge\color{white}\thechapter}}}}% label
{2pt}% horizontal separation between label and title body
{}% before the title body
[]% after the title body

% section format
\titleformat{\section}%
{\normalfont\Large\itshape\color{ColorVariable1}}% format applied to label+text
{\llap{\colorbox{ColorVariable1}{\parbox{1.5cm}{\hfill\color{white}\thesection}}}}% label
{1em}% horizontal separation between label and title body
{}% before the title body
[]% after the title body

% subsection format
\titleformat{\subsection}%
{\normalfont\large\itshape\color{ColorVariable1}}% format applied to label+text
{\llap{\colorbox{ColorVariable1}{\parbox{1.5cm}{\hfill\color{white}\thesubsection}}}}% label
{1em}% horizontal separation between label and title body
{}% before the title body
[]% after the title body

\declaretheoremstyle[
headfont=\color{ColorVariable2}\normalfont\bfseries,
bodyfont=\color{ColorVariable4}\normalfont\itshape,
]{colored-1}

\declaretheoremstyle[
headfont=\color{ColorVariable3}\normalfont\bfseries,
bodyfont=\color{ColorVariable5}\normalfont\itshape,
]{colored-2}

\declaretheorem[
style=colored-1,
name=Problema,
numberwithin=chapter,
%shaded={rulecolor=Lavender,
	%	rulewidth=2pt,
	%	bgcolor={rgb}{1,1,1}}
]{problema}

\declaretheorem[
style=colored-1,
name=Problema Resuelto,
numberwithin=section,
%shaded={rulecolor=Lavender,
	%	rulewidth=2pt,
	%	bgcolor={rgb}{1,1,1}}
]{resuelto}

\declaretheorem[
style=colored-1,
name=Solución,
numbered = no
]{solucion}

\declaretheorem[
style=colored-2,
name= Ejemplo,
numberwithin=section
]{ejemplo}

\declaretheorem[
style=colored-2,
name= Definición,
numbered = no
]{definicion}

\newenvironment{algoritmo}[1]{
	\medskip
	%\begin{framed}
	\bgroup
	\color{ColorVariable2}
	{\textbf{Algoritmo. (#1)}}
	\ttfamily
}{
	\egroup
	%\end{framed}
	\medskip
}

%\newtheorem{teorema}{Teorema}%[chapter]

\declaretheorem[
style=colored-2,
name=Teorema,
numberwithin=chapter,
]{teorema}

%\newtheorem{proposicion}{Proposición}%[chapter]

\declaretheorem[
style=colored-2,
name=Proposición,
numberwithin=chapter,
]{proposicion}

%\newtheorem{observacion}{Observación}%[chapter]

\declaretheorem[
style=colored-2,
name=Observación,
numbered = no
]{observacion}

\newtheorem{axioma}{Axioma}%[chapter]
\newtheorem{sugerencia}{Sugerencia}%[chapter]
\newtheorem{corolario}{Corolario}%[chapter]
\newtheorem{tdv}{Tabla de Verdad}

