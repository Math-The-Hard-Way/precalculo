\section{Fracciones parciales}


 La técnica de fracciones parciales se utiliza para integrar funciones racionales, es decir, aquellas de la forma $$\dfrac{N(x)}{D(x)},$$ donde $N, D$ son polinomios.



 Por simplicidad, supondremos que
 \begin{enumerate}
  \item El coeficiente líder de $D(x)$ es igual a $1.$ 
  \item El grado de $D(x)$ es mayor que el de $N(x).$
 \end{enumerate}
 Sin embargo, ninguna de estas dos condiciones son esenciales.



 \begin{problema}
  \label{ayr:exmp:33.1}
  $$
    \int\dfrac{2x^{3}}{5x^{8}+3x-4}dx=
    \dfrac{1}{5}\int\dfrac{2x^{3}}{x^{8}+\frac{3}{5}x-\frac{4}{5}}
  $$
 \end{problema}




 \begin{problema}
  $$
  \dfrac{2x^{5}+7}{x^{2}+3}=2x^{3}-6x+\dfrac{18x+7}{x^{2}+3}
  $$
 \end{problema}




 \begin{definicion}
  Un polinomio es irreducible si no se puede expresar como el producto de dos polinomios de grado menor. 
 \end{definicion}




	\begin{enumerate}
		\item  Todo polinomio lineal es irreducible
		
		\item 
		Un polinomio cuadrático 
		$$g(x)=ax^{2}+bx+c, \, a\neq0$$ es irreducible si y solo $b^{4}-4ac<0.$
		
	\end{enumerate}



 \begin{problema}
  \label{ayr:exmp:33.3}
  Verifique que 
  \begin{enumerate}
   \item $x^{2}+4$ es irreducible;
   \item $x^{2}+x-4$ es reducible.
  \end{enumerate}

 \end{problema}




 \begin{teorema}
  \label{ayr:thm:33.1}
  Todo polinomio cuyo coeficiente líder sea igual a $1$ se puede expresar como producto de factores lineales, o factores cuadráticos irreducibles.
 \end{teorema}




 \begin{problema}
  \label{ayr:exmp:33.4}
  \begin{enumerate}
   \item $x^{3}-4x=$ 
   \item $x^{3}+4x=$ 
   \item $x^{4}-9=$ 
   \item $x^{3}-3x^{2}-x+3=$
  \end{enumerate}

 \end{problema}



\subsection{Método de Fracciones Parciales}


\subsection{Caso I. $D(x)$ es producto de factores lineales distintos}
 \begin{problema}
  \label{ayr:exmp:33.5}
  Resuelva $$\int \dfrac{dx}{x^{2}-4}$$
 \end{problema}




 \begin{problema}
  \label{ayr:exmp:33.6}
  Resuelva $$\int\dfrac{(x+1)dx}{x^{3}+x^{2}-6x}$$
 \end{problema}




\subsection{Regla General para Caso 1}
 El integrando se representa como una suma de términos de la form $\dfrac{A}{x-a},$ para cada factor $x-a,$ y $A$ una constante por determinar.



\subsection{Caso 2. $D(x)$ es producto de factores lineales repetidos.}
 \begin{problema}
  \label{ayr:exmp:33.7}
  Encuentre $$
  \int\dfrac{(3x+5)dx}{x^{3}-x^{2}-x+1}
  $$
 \end{problema}




 \begin{problema}
  \label{ayr:33.8}
  $$
  \int \dfrac{(x+1)dx}{x^{3}(x-2)^{2}}
  $$
 \end{problema}




\subsection{Regla General para el Caso 2.}
 Para cada factor $x-c$ de multiplicidad $k,$ se utiliza la expresión
 $$
 \dfrac{A_{1}}{x-r}+\dfrac{A_{2}}{(x-r)^{2}}+...+\dfrac{A_{k}}{(x-r)^{k}}.
 $$



\subsection{Caso 3. Factores cuadráticos irreducibles distintos, y lineales repetidos}
 A cada factor irreducible $x^{2}+bx+c$ de $D(x)$ le corresponde el integrando
 $$
 \dfrac{Ax+B}{x^{2}+bx+c}.
 $$



 \begin{problema}
  Encuentre $$
  \int \dfrac{(x-1)dx}{x(x^{2}+1)(x^{2}+2)}
  $$
 \end{problema}




\subsection{Caso IV. Factores cuadráticos irreducibles repetidos}
 
 A cada factor cuadráticos irreducible $x^{2}+bx+c$ de mutiplicidad $k$ le corresponde el integrando
 $$
 \sum_{i=1}^{k}\dfrac{A_{i}x+B_{i}}{(x^{2}+bx+c)^{i}}
 $$
 



 \begin{problema}
  Encuentre $$\int\dfrac{2x^{2}+3}{(x^{2}+1)^{2}}dx.$$
 \end{problema}


