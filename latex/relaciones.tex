\section{Relaciones}


	{Ejemplos de relaciones}
	\begin{itemize}
		\item ``menor que''
		\item ``es paralelo a''
		\item ``es un subconjunto de''
	\end{itemize}
	



	Formalmente, definiremos una relación en t\'erminos de \emph{pares ordenados.}



	\begin{definicion}
		Un \emph{par ordenado} de elementos $a$ y $b,$ donde $a$ es el primer elemento y $b$ es el segundo se denota por $(a,b).$
	\end{definicion}
	



	
	\begin{ax}
		$(a,b)=(c,d)$  si y sólo si $a=c$ y
		$b=d.$
	\end{ax}
	
	
	
	En particular $(a,b)\neq(b,a),$  al menos que $a=b.$
	
	
	
	Esto es muy diferente al caso de un conjunto, dónde el orden es irrelevante:
	$$
	\set{a,b}=\set{b,a}.
	$$


\subsection{Producto de conjuntos}


	Consideremos dos conjuntos arbitrarios $A$ y $B.$ El conjunto de todos los pares ordenadors $(a,b)$ donde $a\in A, b \in B$ es llamado \emph{producto(cartesiano)} de $A$ con $B,$ y se denota por $A \times B,$ es decir,
	$$
	A \times B = \set{(a,b) \mid a \in A, \; b \in B}
	$$



	Podemos construir el producto cartesiano de un conjunto $A$ consigo mismo, y en ese caso denotaremos
	$$A^{2}= A\times A.$$



	\begin{problema}
		Sea $A=\set{x,y}, \, B={0,1}.$ Entonces
		\begin{enumerate}
			\item $A^{2}=\set{(x,x), (x,y), (y,x), (y,y)}$
			\item $A\times B= \set{(x,0), (x,1), (y,0), (y,1)}$
			\item $B\times A= \set{(0,x), (0,y), (1, x), (1,y)}$
			\item $B^2=\set{(0,0), (0,1), (1,0), (1,1)}$
		\end{enumerate}
		
	\end{problema}
	



	\begin{observacion}
		\begin{itemize}
			\item En general, $A\times B \neq B \times A.$
			\item Si \emph{$n(A)$} denota el \emph{número de elementos} en el conjunto $A,$ entonces
			$$
			n(A \times B)= n(A) \cdot n(B).
			$$
		\end{itemize}
		
	\end{observacion}
	




	Sean $A=\set{1,2}$ y $B={a,b,c}.$ Determine $A\times B,$ $B\times A$ y $A^{2},$ y describa gráficamente estos productos.



	\begin{problema}
		$\R^{2}=\R \times \R$ es llamado frecuentemente el \emph{plano Cartesiano.}
	\end{problema}
	



	\begin{definicion}
		Definimos el producto cartesiano de un número finito de conjuntos $A_{1},...,A_{n}$ como
		$$
		\prod_{i=1}^{n} A_{i}= A_{1} \times \cdots \times A_{n}=\set{\left( a_{1},\dots,a_{n} \right)\mid a_{1}\in A_{1}, \dots, a_{n}\in A_{n}}
		$$
	\end{definicion}
	



	\begin{observacion}
		De manera análoga al caso $n=2,$ definiremos
		$$
		A^{n}=\prod_{i=1}^{n}A.
		$$
		
		
		Por ejemplo, $\R^{3}$ denota el espacio tridimensional.
	\end{observacion}
	


\subsection{Relaciones}


	\begin{definicion}
		Sean $A$ y $B$ conjuntos arbitrarios. Una \emph{relación binaria $R$,} o simplemente relación, de $A$ a $B$ es un subconjunto de $A \times B.$
	\end{definicion}
	



	Para cada $(a,b)\in A \times B$ alguna de las siguientes condiciones (pero no ambas) es cierta:
	\begin{enumerate}
		\item $(a,b)\in R;$ en cuyo caso diremos que \emph{$a$ está $R-$relacionado con $b$,} y escribiremos $a \rel{R} b.$
		\item $(a,b)\not\in R;$ en cuyo caso diremos que \emph{$a$ no está $R-$relacionado con $b$,} y escribiremos $a \nrel{R} b.$
	\end{enumerate}
	



	Si $R$ es una relación de $A$ en sím mismo, es decir $R \subset A^{2},$ entonces diremos que $R$ es una \emph{relación en $A$.}



	\begin{definicion}
		Si $R \subset A \times B$ es una relación, el \emph{domino de $R$} es 
		$$
		\dominio{R}=\set{a\in A\mid (a,b)\in R},
		$$ mientras que la \emph{imagen de $R$} es 
		$$
		\imagen{R}=\set{b\in B\mid (a,b)\in R}.
		$$
	\end{definicion}
	


\subsection{Ejemplos}

	Sean $A=\set{1,2,3},$ $B=\subset{x,y,z}$ y $$R=\set{(1,y), (1,z), (3,y)}.$$ Entonce $R$ es una relación de $A$ en $B,$ porque $R \subset A \times B.$
	
	
	Respecto a esta relación, por ejemplo,
	$$
	1\rel{R}y, \; 1\rel{R}z, \; 3\rel{R}y,
	$$ pero 
	$$
	1\nrel{R}x, 2\nrel{R}x, 2\nrel{R}y.
	$$
	
	
	En este caso, $\dominio{R}=\set{1,3}$ e $\imagen{R}=\set{y,z}.$



	La propia inclusión $\subset$ es una relación en una colección de conjuntos $A_{1},...,A_{n}.$ 
	
	Para cualquier par $A_{i}, A_{j}$ en dicha colección $A \subset B$ o $A \not\subset B.$



	Una relación en el conjunto $\Z$ de número enteros es \emph{\texttt{``$m$ divide a $n.$''}}
	
	
	La notación convencional para esta relación es \emph{$m \mid n.$}



	Consideremos el conjunto de lineas $L$ en el plano. La perpendicularidad $\perp$ es una relación en $L.$  De manera similar el paralelismo $\parallel.$



	Sea $A$ cualquier subconjunto. Una relación importante en $A$ es la \emph{igualdad}
	$$
	\set{(a,a) \mid a \in A}
	$$ que usualmente se denota por \emph{$$``=''$$} 
	 En ocasiones, tambi\'en se le llama \emph{entidad} o \emph{diagonal} y se denota por $\triangle_{A},$ o simplemente por $\triangle.$



	Sea $A$ un conjunto arbitrario. Entonces tanto $A\times A$ como $\emptyset$ son subconjuntos de $A \times A,$ y son llamados \emph{relación universal} y \emph{relación vacía,} respectivamente.



	{Relación inversa}
	
	Sea $R$ una relación de $A$ en $B.$ La \emph{relación inversa} de $R,$ denotada por \emph{$R^{-1}$,} es la relación de $B$ en $A$ que consiste en todos aquellos pares que al invertirlos, pertenecen a $R.$ 
	
	En otras palabras
	$$
	R^{-1}=\set{(b,a)\mid (a,b)\in R}.
	$$



	\begin{problema}
		Sea $A=\set{1,2,3}, B=\set{x,y,z}$ y $R=\set{(1,y),(1,z),(3,y)}.$ Entonces
		$$
		R^{-1}=\set{(y,1), (z,1), (y,3)}.
		$$
	\end{problema}



	\begin{observacion}
		\begin{itemize}
			\item $\left( R^{-1} \right)^{-1}=R.$
			\item $\dominio{R^{-1}}=\imagen{R}$
			\item $\imagen{R^{-1}}=\dominio{R}$
		\end{itemize}
	\end{observacion}




\subsection{Composición de Relaciones}

	Sean $A,B,C$ conjuntos arbitrarios, $R$ una relación de $A$ en $B$ y $S$ una relación de $B$ en $S.$  Entonces podemos definir una nueva relación de $A$ en $C$ denotada por \emph{$RS$:}
	\begin{center}
		$a\rel{{RS}}c$ si para alguna $b \in B,$ $a\rel{R}b$ y $b\rel{S}c.$
	\end{center} 



	Esto es
	$$
	RS=\set{(a,c)\mid \exists b\in B: (a,b)\in R, (b,c)\in S}
	$$



	Supongamos que $R$ es una relación en $A.$ Entonces, definimos $R^{n}$ de manera recursiva
	$$
	R^{1}=
	\begin{cases}
		R & n=1 \\
		R^{n-1}R & n>1
	\end{cases}
	$$




	



	\begin{teorema}
		Supongamos que $R$ es uan relación de $A$ en $B,$ y $S$ una relación de $B$ en $C.$ Entonces
		$$
		(RS)T=R(ST).
		$$
	\end{teorema}
	


\subsection{Tipos de relaciones}

\paragraph{Relaciones reflexivas}


	
	Una relación $R$ es un conjunto $A$ es \emph{reflexiva} si $a\rel{R}a$ para todo $a\in A$, \, es decir, $\forall a \in A: (a,a)\in \R.$ 
	



\paragraph{Relaciones sim\'etricas y antisim\'etricas}


	Una relación $R$ en un conjunto $A$ es sim\'etrica si: Siempre que $a\rel{R}b,$ entonces $b\rel{R}a.$  En otras palabras, 
	$$
	(a,b)\in \R \onlyif (b,a)\in \R.
	$$





	Una relación $R$ en un conjunto $A$ es antisim\'etrica si: Siempre que $a\rel{R}b$ y $b\rel{R}a$ entonces $a=b.$  En otras palabras, 
	$$
	a\neq b, a\rel{R}b \onlyif b\nrel{R}a.
	$$
	


	\begin{observacion}
		Las propiedades de simetría y antisimetría no son excluyentes una de la otra. 
		
		Por ejemplo, la relación $$R=\set{(1,3),(3,1),(2,3)}$$ no es sim\'etrica ni antisim\'etrica. 
		
		Por otro lado, la relación $$S=\set{(1,1),(2,2)}$$ es tanto sim\'etrica como antisim\'etrica.
	\end{observacion}
	


\paragraph{Relación transitiva}


	Una relación $R$ en un conjunto $A$ es transitiva si: Siempre que $a\rel{R}b$ y $b\rel{R}c,$ entonces $a\rel{R}c.$  En otras palabras, 
	$$
	(a,b)\in R, (b,c)\in R \onlyif (a,c)\in \R.
	$$
	
	


 \paragraph{Propiedades de cerradura}
 
 
 Consideremos un conjunto dado $A$ y la colección de todas las relaciones en $A,$ y sea $P$ una propiedad en la colección de tales relaciones, por ejemplo, la simetría o la transitividad.
 
 
 Si una relación satisface la propiedad $P,$ diremos que es una $P-$relación.
 

\subsection{Relaciones de Equivalencia}

	Considere un conjunto no-vacío $S.$ Una relación $R$ en $S$ es una \emph{relación de equivalencia} si $R$ es reflexiva, sim\'etrica y transitiva.



	En otras palabras, $R$ es una \emph{relación de equivalencia} en $S$ si satisface las siguientes propiedades:
	\begin{enumerate}
		\item Para cada $a\in S:$ $a\rel{R}a;$
		\item si $a\rel{R}b,$ entonces $b\rel{R}a;$
		\item si $a\rel{R}b,$ $b\rel{R}c,$ entonces $a\rel{R}c.$
	\end{enumerate}
	



	La idea general detras de una relación de equivalencia que es una clasificación de objetos que son en cierto sentido \emph{similares.} 
	
	
	Por ejemplo, la relación \emph{$=$} de igualdad en cualquier conjunto $S$ es una relación de equivalencia, porque...



	\begin{problema}
		\label{lip:exmp:2.12.a}
		Sea $L$ el conjunto de líneas en el plano cartesiano y $T$ el conjunto de triangulos en el mismo plano.
		
		\begin{enumerate}
			\item La relación de paralelidad es una relación de equivalencia en $L;$ 
			\item La relación de congruencia o la de similaridad son relaciones de equivalencia en $T.$
		\end{enumerate}
		
	\end{problema}
	


	\label{lip:exmp:2.12.c}
	Sea $m$ un entero positivo fijo. Dos enteros $a,b$ son llamados \emph{congruentes módulo $m,$} si $m$ divide la diferencia $a-b,$ y en tal caso escribimos:
	$$
	a\equiv b \mod m.
	$$
	
	
	Por ejemplo $11\equiv 3 \mod 4$ y $22\equiv 6 \mod 4.$
	
	
	La relación de congruencia módulo $m$ es un relación de equivalencia. 



\subsection{Particiones y clases de equivalencia}


	Una paritición $P$ de un conjunto no-vacío $S$ es una colección $\set{A_{j}}$de subconjuntos no-vaciós de $S$ con las siguientes propiedades de que cada $a\in S$ pertenece a uno y solo uno de los conjunto $A_{j}$ de la partición.  
	
	En otras palabras,
	\begin{enumerate}
		\item Cada $a\in S$ pertenece a algún $A_{j};$
		\item si $A_{i}\neq A_{j},$ entonces $A_{j}\cap A_{j}=\emptyset.$
		
	\end{enumerate}
	
	
	De manera equivalente, una partición $P$ de $S$ es una subdivisión de $S$ en conjuntos disjuntos no vacíos $A_{j}$ tal que $$S= \sqcup_{j} A_{j}.$$



	Supontamos que $R$ es una relación de equivalencia en el conjunto $S.$ Para cada $a\in S,$ denotemos por $[a]$ el conjunto de elementos de $S$ tales que están $R-$relacionados con $a.$ 
	
	
	En otras palabras,
	$$
	[a]=\set{x \in S \mid (a,x)\in R}.
	$$



	La colección de clases de equivalencia de elementos de $S$ bajo una relación de de equivalencias $R$ se denota por $S/R,$ es decir,
	$$
	S/R=\set{[a] \mid a \in S}.
	$$
	 
	
	Diremos que $S/R$ es el conjunto cociente de $S$ por $R.$



	\begin{teorema}
		\label{lip:thm:2.6}
		Sea \emph{$R$ una relación de equivalencia en $S.$} Entonces \emph{S/R es una partición de $S.$} 
		De manera especifica:
		\begin{enumerate}
			\item Para cada $a \in S:$  $a\in [a];$
			\item $[a]=[b]$ si y solo si $(a,b)\in R;$
			\item Si $[a]\neq [b],$ entonces $[a]$ y $[b]$ son conjuntos disjuntos. 
		\end{enumerate}
		
		
		De manera inversa, dada una partición $P=\set{A_{j}}$ de conjuntos $S,$ existe una relación $R$ en $S$ tal que los conjuntos $A_{j}$ son las clases de de equivalencia de $R.$
	\end{teorema}
	



	\begin{problema}
		\label{lip:exmp:2.13.b}
		Sea $R=\set{(1,1),(1,2),(2,1),(2,2),(3,3)}$ en $S=\set{1,2,3}.$ Demuestre que $R$ es una relación de equivalencia y calcule $S/R.$
	\end{problema}
	






\paragraph{Relaciones de orden parcial}


	Una relación $R$ en un conjunto $S$ es llamada \emph{orden parcial} de $S$ en $R$ si es reflexiva, antisim\'etrica y transitiva. 
	
	Un conjunto $S$ con un orden parcial $R$ es llamado \emph{conjunto parcialmente ordenado} o \emph{poset.}