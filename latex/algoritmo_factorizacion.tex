
%303
\section{Algoritmo de factorización}
\subsection{Diferencias de potencias}


	\begin{proposicion}
		\[
			\label{pow:diff}
			\tag{dP}
			x^{n}-c^{n}=(x-c)\left( x^{n-1}+x_{n-2}c+...+xc^{n-2}+c^{n-1} \right)
		\]
		
	\end{proposicion}
	



	\begin{problema}
		\begin{align*}
			x^{2}-121&= \\
			x^{3}-27&= \\
			x^{4}-256&=
		\end{align*}		
	\end{problema}
	



	\begin{problema} 
		\begin{align*} 
			x^{2}-c^{2}
			&=(x-c)\left( x^{1}+c^{1} \right) \\
			&=(x-c)(x+c)  \\ 
			x^{3}-c^{3}
			&=(x-c)\left( x^{2}+x^{1}c^{1}+c^{2} \right)  \\
			& = (x-c)\left( x^{2}+cx+c^{2} \right) \\
			x^{4}-c^{4}
			&=(x-c)\left( x^{3}+x^{2}c^{1}+x^{1}c^{1}+c^{3} \right) \\
			&=(x-c)\left( x^{3}+cx^{2}+c^{2}x+c^{3} \right)
		\end{align*}
		
	\end{problema}
	



	El segundo factor en el lado derecho de \eqref{pow:diff} se puede reescribir de la siguiente manera:  
	\begin{align*}
		\nonumber
		x^{n-1}+x^{n-2}c+...+xc^{n-2}+c^{n-1} \\
		\nonumber 
		=x^{n-1}c^{0}+x^{n-2}c^{1}+...+x^{1}c^{n-2}+x^{0}c^{n-1} \\
		\label{sym:sum}
		\tag{pS}
		=\sum_{i+j=n-1}x^{i}c^{j}
		=:S^{n-1}_{x,c}
	\end{align*}
	
	donde \emph{$\sum_{i+j=M}$} denota la suma la suma sobre todas las parejas $i,j$ de números naturales, cuya suma sea igual a $M.$
	

%%%%%%%%%%%%%%%%%%%%%
{}
	
	Diremos que \emph{$S_{x,c}^{M}$} es el \textcolor{blue}{polinomio sim\'etrico} de grado $M$ (para $x,c$).



	\begin{problema}
		Calcule los siguientes polinomios sim\'etricos
		\begin{flalign*}
			S^{1}_{x,11}&=   x+11 &\\
			S^{2}_{x,3}&=  x^2+3x+9 &\\
			S^{3}_{x,4}&=  x^3+4x^2+16x+64
		\end{flalign*}
	\end{problema}
	


\subsection{Divisores de un polinomio}

	\begin{definicion}
		Decimos que un polinomio $D(x)$ divide a otro polinomio $P(x)$ si existe un tercer polinomio $Q(x)$ tal que $D(x)Q(x)=P(x).$
		
		
		En tal caso decimos que $D(x)$ divide a $P(x)$ y escribimos $D(x) \mid P(x).$ Al polinomio $Q(x)$ se le llama \emph{polinomio cociente.}
	\end{definicion}
	



	\begin{teorema}
		Un número $x=c$ es un cero de $P(x)$ si y solo si $(x-c)$ divide a $P(x).$
		
		
		Diremos que $D_{c}(x)=(x-c)$ es el \emph{divisor asociado} a $x=c.$
	\end{teorema}
	



	\begin{algoritmo}{Factorizaci\'on de un divisor asociado}
		Supongamos que $x=c$ es un cero del polinomio $P(x)=a_{n}x^{n}+...+a_{1}x+a_{0}.$ %Entonces
		\begin{enumerate}
			\item Rescribimos explicitamente
			$P(x)=P(x)-P(c)$  
			\item Factorizamos cada coeficiente
			$$P(x)=a_{n}\left( x^{n}-c^{n} \right)+...+a_{1}(x-c)$$
			\item Aplicamos diferencias de cuadrados en cada t\'ermino
			$$P(x)=a_{n}(x-c)S^{n-1}_{x,c}+...+a_{1}(x-c)$$
			\item Factorizamos $D_{c}(x)=x-c$
			$$P(x)=\left( x-c \right)\left( a_{n}S^{n-1}_{x,c}+...+a_{1} \right)$$
		\end{enumerate}
		
	\end{algoritmo}
	

%%%%%%%%%%%%%%%%%%%%%% 

	\begin{problema} Para cada uno de los siguientes polinomios, factorice los divisores asociados a cada uno de sus ceros racionales tantas veces como sea posible, utilizando diferencias de potencias:
		
		\begin{enumerate}
			\item $x^{3}+3x^{2}-4={\left(x + 2\right)}^{2} {\left(x - 1\right)}
			$
			\item $x^{4}-5x^{2}+4={\left(x + 2\right)} {\left(x + 1\right)} {\left(x - 1\right)} {\left(x - 2\right)}
			$
		\end{enumerate}
		
		
	\end{problema}

%%%%%%%%%%%%%%%%%%%%%% 

	\begin{algoritmo}{Encontrar los ceros racionales de un polinomio}
		\begin{enumerate}
			\item \emph{Enlistar los posibles ceros.} Enliste los posibles ceros racionales usando el teorema de los ceros racionales.
			\item \emph{Dividir.} Use la divisi\'on sint\'etica para evaluar el polinimio en cada uno de los candidatos para ceros racionales que encontr\'o en el paso anterior. Cuando el residuo es $0,$ observe el cociente que obtuvo.
			\item \emph{Repetir.} Repita los pasos anteriores para el cociente. Pare cuando llegue al cociente que no tenga ceros racionales.
		\end{enumerate}
		
	\end{algoritmo}



	\begin{problema} Para cada uno de los siguientes polinomios, factorice los divisores asociados a cada uno de sus ceros racionales tantas veces como sea posible:
		
		\begin{enumerate}
			\item $x^{3}+3x^{2}-4={\left(x + 2\right)}^{2} {\left(x - 1\right)}
			$
			\item $x^{4}-5x^{2}+4={\left(x + 2\right)} {\left(x + 1\right)} {\left(x - 1\right)} {\left(x - 2\right)}
			$
		\end{enumerate}
		
		
	\end{problema}


\subsection{Raíces irracionales}

%%%%%%%%%%%%%%%%%%%%%

Un polinomio cuadrático 
\begin{align*}
	p(x)=ax^{2}+bx+c, a\neq 0
\end{align*}
tiene raíces $r_{1}$ y $r_{2}$ si y solo si 
\begin{align*}
p(x) = a\left( x-r_{1} \right)\left( x-r_{2} \right)
\end{align*}


%%%%%%%%%%%%%%%%%%%%%
{Fórmula general}
	\begin{proposicion}
	Las soluciones de la ecuación 
	\begin{align*}
		ax^{2}+bx+c=0, a\neq 0
\end{align*}
están dadas por la fórmula 
\begin{align*}
\begin{cases}
D = b^{2}-4ac \\
r = \dfrac{-b\pm\sqrt{D}}{2a}
\end{cases}
\end{align*}

	\end{proposicion}


%%%%%%%%%%%%%%%%%%%%%
{Discriminante}
El número $D=b^{2}-4ac$ se llama \emph{discriminante} del polinomio cuadrático $p(x)= ax^{2}+bx+c$.

%%%%%%%%%%%%%%%%%%%%%


\begin{corolario}
\begin{enumerate}
\item Si $D>0$, entonces $p(x)=0$ tiene exactamente dos raíces reales y diferentes.
\item Si $D=0$, entonces $p(x)=0$ tiene una única raíz real de multiplicidad $2$.
\item Si $D<0$, entonces $p(x)=0$ tiene un par de raíces complejas conjugadas.
\end{enumerate}

\end{corolario}


%%%%%%%%%%%%%%%%%%%%%

\begin{problema}
Factorice completamente el polinomio
\begin{align*}
P(x) = x^{4}-5x^{3}-5x^{2}+23x+10
\end{align*}
\end{problema}


