
%201
\section{Reducci\'on de t\'erminos semejantes}

 En el álgebra, representamos cantidades desconocidas por s\'imbolos;  generalmente son letras como $x,y,z,$ pero \emph{no debe olvidarse que respresentan números.}



	Para representar una multiplicaci\'on iterada, usamos el s\'imbolo de potencia
	$$
	x^{n}=\underbrace{x\cdot\cdots \cdot x}_{n\texttt{-veces}};
	$$
	al número $x$ le llamamos base y al número $n$ le llamamos potencia.



	\begin{problema}
		\begin{enumerate}
			\item $3^{2}=3\cdot3=9$
			\item $5^{3}=5\cdot5\cdot5=125$
			\item $2^{4}=2\cdot2\cdot2\cdot2=16$
		\end{enumerate}
		
	\end{problema}
	



	\begin{problema}
		\begin{enumerate}
			\item $x^{2}=x\cdot x$
			\item $x^{3}=x\cdot x\cdot x$
			\item $x^{4}=x\cdot x\cdot x\cdot x$
			\item $\cdots$
		\end{enumerate}
		
	\end{problema}




	\begin{observacion}
		Observe que $x^{1}=x;$ mientras que, por convenci\'on, $x^{0}=1.$
	\end{observacion}
	



	Cuando multiplicamos $x^{n}$ por un número diferente de $x:$
	$$ax^{n}=a\cdot\underbrace{x\cdot\cdots *x},$$ diremos que $a$ es el coeficiente de $x^{n}.$ 



	A un número escrito en la forma $ax^{n}$ se le llama \emph{monomio;} y diremos que dos monomios son semejantes si tienen \emph{exactamente} la misma base a la misma potencia.



	\begin{problema}
		Determine cual de los siguientes monomios es semejante a $2x^{3}:$
		\begin{enumerate}
			\item $3x^{3};$
			\item $2x^{2};$
			\item $2y^{3};$
		\end{enumerate}
		
	\end{problema}
	



	Dos t\'erminos semejantes pueden reducirse 
	$$\begin{cases}
		ax^{n}+bx^{n}=\left( a+b \right)x^{n}\\
		ax^{n}-bx^{n}=\left( a-b \right)x^{n}
	\end{cases}
	$$



	\begin{problema}
		Reduzca los siguientes t\'erminos semejantes, escribiendo el coeficiente en forma irreducible.
		\begin{enumerate}
			\item $\left( 4n^{2}+5n+3 \right)+\left( -3n^{2}-2n \right)$
			\item $\left( -7a^{3}-a^{2}-5a-2 \right)+\left( a^{3}+a^{2}-4a+7 \right)$
			\item \begin{eqnarray*}
				\left(5u^{5}-3u^{4}+2u^{3}-5u^{2}+7u-7\right)+ \\
				\left( \dfrac{2u^{5}}{3}+\dfrac{3u^{4}}{2}-\dfrac{2u^{2}}{3}+\dfrac{5u}{6}+\dfrac{5}{4} \right).
			\end{eqnarray*}
			
		\end{enumerate}
		
	\end{problema}
	



	\begin{problema}
		Reduzca los siguientes t\'erminos semejantes, escribiendo el coeficiente en forma irreducible.
		\begin{enumerate}
			\item $\left( 4n^{2}+5n+3 \right)-\left( -3n^{2}-2n \right)$
			\item $\left( -7a^{3}-a^{2}-5a-2 \right)-\left( a^{3}+a^{2}-4a+7 \right)$
			\item \begin{eqnarray*}
				\left(5u^{5}-3u^{4}+2u^{3}-5u^{2}+7u-7\right)- \\
				\left( \dfrac{2u^{5}}{3}+\dfrac{3u^{4}}{2}-\dfrac{2u^{2}}{3}+\dfrac{5u}{6}+\dfrac{5}{4} \right).
			\end{eqnarray*}
			
		\end{enumerate}
		
	\end{problema}
	



	\begin{observacion}
		A la suma de dos o más monomios se le conoce como \emph{polinomio}.
		
		Por ejemplo $x^{2}-2x+1$ o $x^{2}-2xy+y^{2}.$
		
	\end{observacion}
	

