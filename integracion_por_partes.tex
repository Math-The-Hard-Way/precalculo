
\section{Integraci\'on por partes}
A partir de la regla del producto
$$
\dfrac{d}{dx}\left( uv \right)=u\dfrac{dv}{dx}+v\dfrac{du}{dx},
$$ se deduce la f\'ormula de integraci\'on por partes:
\[
 \label{int:partes}
 \int udv=uv-\int vdu
\]




	Para elegir $u$, podemos seguir la regla empírica \textbf{LIATE}: 
	\begin{itemize}
		\item \textbf{L}ogaritmos 
		\item \textbf{I}nversas trigonométricas 
		\item \textbf{A}lgebraicas 
		\item \textbf{T}rigonométricas 
		\item \textbf{E}xponenciales 
	\end{itemize}



 \begin{problema}
  Encuentre $$\int x \ln(x) dx.$$
 \end{problema}




 \begin{problema}
  Encuentre $$\int x e^{x} dx.$$
 \end{problema}




 \begin{problema}
  Encuentre $$\int e^{x}\cos(x) dx.$$
 \end{problema}



\subsection{Ejemplos Resueltos}


\begin{problema} Por medio de integraci\'on por partes, encuentre la siguiente integral indefinida
 $$
 \displaystyle \int x^{3}e^{x^{2}} dx
 $$
\end{problema}





\begin{problema} Por medio de integraci\'on por partes, encuentre la siguiente integral indefinida
 $$
 \displaystyle \int \ln\left( x^{2}+2 \right)dx
 $$
\end{problema}




\begin{problema} Por medio de integraci\'on por partes, encuentre la siguiente integral indefinida
 $$
 \displaystyle \int \ln(x) dx
 $$
\end{problema}




\begin{problema} Por medio de integraci\'on por partes, encuentre la siguiente integral indefinida
 $$
 \displaystyle \int x\sin(x) dx
 $$
\end{problema}




\begin{problema} Por medio de integraci\'on por partes, encuentre la siguiente integral indefinida
 $$
 \displaystyle \int x^{2}\ln(x) dx
 $$
\end{problema}




\begin{problema} Por medio de integraci\'on por partes, encuentre la siguiente integral indefinida
 $$
 \displaystyle \int \sin^{-1}(x) dx
 $$
\end{problema}




\begin{problema} Por medio de integraci\'on por partes, encuentre la siguiente integral indefinida
 $$
 \displaystyle \int \tan^{-1}(x) dx
 $$
\end{problema}




\begin{problema} Por medio de integraci\'on por partes, encuentre la siguiente integral indefinida
 $$
 \displaystyle \int \sec^{3}(x) dx
 $$
\end{problema}




\begin{problema} Por medio de integraci\'on por partes, encuentre la siguiente integral indefinida
 $$
 \displaystyle \int x^{2}\sin(x) dx
 $$
\end{problema}




\begin{problema} Por medio de integraci\'on por partes, encuentre la siguiente integral indefinida
 $$
 \displaystyle \int x^{3}e^{2x} dx
 $$
\end{problema}




\begin{problema}
 Deduzca la siguiente \emph{f\'ormula de reducci\'on}
\[
 \label{reduccion}
 \tag{FR}
 \begin{split}
   \displaystyle \int \sin^{m}(x)dx= & -\dfrac{\sin^{m-1}(x)\cdot \cos(x)}{m}\\
 &+\dfrac{m-1}{m}\int \sin^{m-2}(x)dx.
 \end{split}
\]

\end{problema}




 \begin{problema}
  Aplique la formula de reducci\'on \eqref{reduccion} para hallar 
  $$
  \displaystyle \int \sin^{2}(x)dx.
  $$
 \end{problema}




  \begin{problema}
  Aplique la formula de reducci\'on \eqref{reduccion} para hallar 
  $$
  \displaystyle \int \sin^{3}(x)dx.
  $$
 \end{problema}
%
%
%\subsection{Técnicas de integaci\'on trigonométrica}
%
%\subsection{Integrados trigonométricos}
%
%
% \subparagraph{Caso 1}
% Considérense las integrales de la forma 
% $$
% \int \sin^{k}(x)cos^{n}(x)dx, 
% $$ con $k,n$ enteros no negativos.
%
%
%
% \subparagraph{Tipo 1.1}
% Si Al menos uno de los números $k,n$ es impar, entonces podemos escoger $u=\cos(x)$ o $u=\sin(x).$
%
%
%
% \begin{problema}
%  \label{ayr:32.1}
%  $$
%  \int \sin^{3}(x)\cos^{2}(x)dx.
%  $$
% \end{problema}
%
%
%
%
% \begin{problema}
% \label{ayr:32.2}
%  $$
%  \int sin^{4}(x)cos^{7}(x)dx
%  $$
% \end{problema}
%
%
%
%
% \begin{problema}
%  \label{ayr:32.3}
%  $$
%  \int sin^{5}(x)dx
%  $$
% \end{problema}
%
%
%
%
% \subparagraph{Tipo 1.2}
% Si ambas potencias $k,n$ son pares. Entonces utilizaremos las identidades
% $$\begin{cases}
%    cos^{2}(x)=\dfrac{1+cos(2x)}{2}\\
%    sin^{2}(x)=\dfrac{1-cos(2x)}{2}
%   \end{cases}
%$$
%
%
%
% \begin{problema}
%  \label{ayr:32.4}
%  $$
%  \int cos^{2}(x)sin^{4}(x)dx
%  $$
% \end{problema}
%
%
%
%
% \subparagraph{Caso 2}
% Consideraremos integrales de la forma 
% $$
% \int tan^{k}(x)sec^{n}(x)dx.
% $$
% y utilizaremos la identidad
%$$
%sec^{2}(x)=1+tan^{2}(x).
%$$
% 
% 
% 
%  \subparagraph{Tipo 2.1} Si $n$ es par, entonces se sustituye $u=\tan(x).$
% 
%
% 
% 
%  \begin{problema}
%   \label{ayr:32.5}
%   $$
%   \int tan^{2}(x)sec^{4}(x)dx
%   $$
%  \end{problema}
%
% 
%
% 
%  \subparagraph{Tipo 2.2} Si $n,k$ son impares, se sustituye $u=sec(x).$
% 
%
% 
%
% \begin{problema}
%  \label{ayr:32.6}
%  $$
%  \int tan^{3}(x)sec(x)dx.
%  $$
% \end{problema}
%
%
%
%
%\subparagraph{Tipo 2.3}
%Si $n$ es impar y $k$ par, reducimos a los casos anteriores y utilizaremos la f\'ormula
%$$
%\displaystyle \int \sec(x)dx= \ln\abs{\tan(x)+\sec(x)}
%$$
%
%
%
%\begin{problema}
% $$
% \int \tan^{2}(x)\sec(x)dx=
% $$
%\end{problema}
%
%
%
% \subparagraph{Caso 3} Consideremos ahora integrales de la forma $\int f(Ax)g(Bx)dx,$ donde $f,g$ pueden ser o bien $sin$ o bien $cos.$ 
%
%
%
%
% Necesitaremos las identidades
% \begin{align*}
%  sin(Ax)cos(Bx)&=\dfrac{1}{2}\left( sin\left( (A+B)x \right)+sin\left( (A-B)x \right) \right)\\
%  sin(Ax)sin(Bx)&=\dfrac{1}{2}\left( cos\left( (A-B)x \right)-cos\left( (A+B)x \right) \right)\\
%  cos(Ax)cos(Bx)&=\dfrac{1}{2}\left( cos\left( (A-B)x \right)+cos\left( (A+B)x \right) \right)
% \end{align*}
%
%
%
% \begin{problema}
%  \label{ayr:32.7}
%  $$
%  \int sin(7x)cos(3x)
%  $$
% \end{problema}
%
%
%
%
% \begin{problema}
%  \label{ayr:exmp:32.8}
%  $$
%  \int sin(7x)cos(3x)
%  $$
% \end{problema}
%
%
%
%
% \begin{problema}
%  \label{ayr:32.9}
%  $$
%  \int sin(7x)sin(3x)
%  $$
% \end{problema}
%
%
%
%
% \begin{problema}
%  \label{ayr:32.10}
%  $$
%  \int cos(7x)cos(3x)
%  $$
% \end{problema}
%
%
%
%\subsection{Sustituci\'on trigonométrica}
%
%
%Existen tres principales tipos de sustituci\'on trigonométrica. Introduciremos cada uno por medio de ejemplos típicos.
%
%
%
%\begin{problema}
% \label{ayr:exmp:32.11}
% Encuentre
% $$
% \displaystyle \int \dfrac{dx}{x^{2}\sqrt{4+x^{2}}}
% $$
%\end{problema}
%
%
%
%% 
%%  \subparagraph{Estrategia I}
%%  \begin{center}
%% %\Huge 
%% Si $\sqrt{a^{2}+x^{2}}$ aparece en el integrando, intente $x=a\tan(\theta)$
%% \end{center}
%%  
%% 
%
%%[c]
%%\begin{center}
% \subparagraph{Estrategia I}
% 
% Si $\sqrt{a^{2}+x^{2}}$ aparece en el integrando, intente $x=a\tan(\theta)$
%
%%\end{center}
%
%
%
%\begin{problema}
% \label{ayr:exmp:32.12}
% Encuentre $$
% \displaystyle \int \dfrac{dx}{x^{2}\sqrt{9-x^{2}}}
% $$
%\end{problema}
%
%
%
%%[c]
%\subparagraph{Estrategia II}
%
% Si $\sqrt{a^{2}-x^{2}}$ aparece en un integrando, trate con la sustituci\'on $x=a\sin(\theta).$
%
%
%
%
%
%\begin{problema}
%\label{ayr:exmp:32.13}
%Encuentre 
%$$
%\displaystyle \int \dfrac{x^{2}}{\sqrt{x^{2}-4}}dx.
%$$
%\end{problema}
%
%
%
%\subsection{Estrategia III}
% Si $\sqrt{x^{2}-a^{2}}$ aparece en un integrando, trate con la sustituci\'on $x=a\sec{\theta}.$
%
%
%
%\section{Fracciones parciales}
%
%
% La técnica de fracciones parciales se utiliza para integrar funciones racionales, es decir, aquellas de la forma $$\dfrac{N(x)}{D(x)},$$ donde $N, D$ son polinomios.
%
%
%
% Por simplicidad, supondremos que
% \begin{enumerate}
%  \item El coeficiente líder de $D(x)$ es igual a $1.$
%  \item El grado de $D(x)$ es mayor que el de $N(x).$
% \end{enumerate}
%Sin embargo, ninguna de estas dos condiciones son esenciales.
%
%
%
% \begin{problema}
%  \label{ayr:exmp:33.1}
%  $$
%    \int\dfrac{2x^{3}}{5x^{8}+3x-4}dx=
%    \dfrac{1}{5}\int\dfrac{2x^{3}}{x^{8}+\frac{3}{5}x-\frac{4}{5}}
%  $$
% \end{problema}
%
%
%
%
% \begin{problema}
%  $$
%  \dfrac{2x^{5}+7}{x^{2}+3}=2x^{3}-6x+\dfrac{18x+7}{x^{2}+3}
%  $$
% \end{problema}
%
%
%
%
% \begin{definicion}
%  Un polinomio es irreducible si no se puede expresar como el producto de dos polinomios de grado menor. 
% \end{definicion}
%
%
%
%
% Todo polinomio lineal es irreducible
%
%
%Un polinomio cuadrático 
% $$g(x)=ax^{2}+bx+c, \, a\neq0$$ es irreducible si y solo $b^{4}-4ac<0.$
%
%
% \begin{problema}
%  \label{ayr:exmp:33.3}
%  Verifique que 
%  \begin{enumerate}
%   \item $x^{2}+4$ es irreducible;
%   \item $x^{2}+x-4$ es reducible.
%  \end{enumerate}
%
% \end{problema}
%
%
%
%
% \begin{teorema}
%  \label{ayr:thm:33.1}
%  Todo polinomio cuyo coeficiente líder sea igual a $1$ se puede expresar como producto de factores lineales, o factores cuadráticos irreducibles.
% \end{teorema}
%
%
%
%
% \begin{problema}
%  \label{ayr:exmp:33.4}
%  \begin{enumerate}
%   \item $x^{3}-4x=$ 
%   \item $x^{3}+4x=$ 
%   \item $x^{4}-9=$ 
%   \item $x^{3}-3x^{2}-x+3=$
%  \end{enumerate}
%
% \end{problema}
%
%
%
%\subsection{Método de Fracciones Parciales}
%
%
% \subsection{Caso I. $D(x)$ es producto de factores lineales distintos}
% \begin{problema}
%  \label{ayr:exmp:33.5}
%  Resuelva $$\int \dfrac{dx}{x^{2}-4}$$
% \end{problema}
%
%
%
%
% \begin{problema}
%  \label{ayr:exmp:33.6}
%  Resuelva $$\int\dfrac{(x+1)dx}{x^{3}+x^{2}-6x}$$
% \end{problema}
%
%
%
%
% \subsection{Regla General para Caso 1}
% El integrando se representa como una suma de términos de la form $\dfrac{A}{x-a},$ para cada factor $x-a,$ y $A$ una constante por determinar.
%
%
%
% \subsection{Caso 2. $D(x)$ es producto de factores lineales repetidos.}
% \begin{problema}
%  \label{ayr:exmp:33.7}
%  Encuentre $$
%  \int\dfrac{(3x+5)dx}{x^{3}-x^{2}-x+1}
%  $$
% \end{problema}
%
%
%
%
% \begin{problema}
%  \label{ayr:33.8}
%  $$
%  \int \dfrac{(x+1)dx}{x^{3}(x-2)^{2}}
%  $$
% \end{problema}
%
%
%
%
% \subsection{Regla General para el Caso 2.}
% Para cada factor $x-c$ de multiplicidad $k,$ se utiliza la expresi\'on
% $$
% \dfrac{A_{1}}{x-r}+\dfrac{A_{2}}{(x-r)^{2}}+...+\dfrac{A_{k}}{(x-r)^{k}}.
% $$
%
%
%
% \subsection{Caso 3. Factores cuadráticos irreducibles distintos, y lineales repetidos}
% A cada factor irreducible $x^{2}+bx+c$ de $D(x)$ le corresponde el integrando
% $$
% \dfrac{Ax+B}{x^{2}+bx+c}.
% $$
%
%
%
% \begin{problema}
%  Encuentre $$
%  \int \dfrac{(x-1)dx}{x(x^{2}+1)(x^{2}+2)}
%  $$
% \end{problema}
%
%
%
%
% \subsection{Caso IV. Factores cuadráticos irreusibles repetidos}
% 
% A cada factor cuadráticos irreducible $x^{2}+bx+c$ de mutiplicidad $k$ le corresponde el integrando
% $$
% \sum_{i=1}^{k}\dfrac{A_{i}x+B_{i}}{(x^{2}+bx+c)^{i}}
% $$
% 
%
%
%
% \begin{problema}
%  Encuentre $$\int\dfrac{2x^{2}+3}{(x^{2}+1)^{2}}dx.$$
% \end{problema}
%
%
%
%\section{Sustituciones misceláneas}
%
%{Caso I}
%Supondremos que en una funci\'on racional, una variable se reemplaza por alguno de los siguientes radicales
%\begin{enumerate}
% \item $\sqrt[n]{ax+b}.$ Entonces, la sustituci\'on $ax+b=z^{n}$ producirá una funci\'on racional.
% \item $\sqrt{q+px+x^2}.$ Aquí la sustituci\'on $q+px+x^{2}=(z-x)^2$ resultará en una funci\'on racional.
% \item $\sqrt{q+px-x^{2}}=\sqrt{(\a+x)(\b-x)}.$ En este caso, la sustituci\'on $q+px-x^{2}=(\a+x)^{2}z^{2}$ producirá una funci\'on racional. 
%\end{enumerate}
%
%
%{Caso II}
%Supondremos que en una funci\'on raiconal, algunas variables se reemplazan por $\sin(x),$ $\cos(x)$ o ambas. Entonces, la sustituci\'on $x=2\tan^{-1}(z)$ resultará en una integral de una funci\'on racional de $z.$ 
%
%
%Esto es porque 
%\[
% \label{ayr:34.1}
% \sin(x)=\dfrac{2z}{1+z^{2}}, \; \cos(x)=\dfrac{1-z^2}{1+z^2}, \;
% dx=\dfrac{2}{1+z^2}dz
%\]
%
%
%
%
%Resuelva $$\displaystyle \int \dfrac{dx}{x\sqrt{1-x}}$$
%
%
%
%Resuelva $$\displaystyle \int \dfrac{dx}{(x-2)\sqrt{x+2}}$$
%
%
%
%Resuelva $$\displaystyle \int \dfrac{dx}{x^{1/2}-x^{1/4}}$$
%
%
%
%Resuelva $$\displaystyle \int \dfrac{dx}{x\sqrt{x^2+x+2}}$$
%
%
%
%Resuelva $$\displaystyle \int \dfrac{xdx}{\left( 5-4x-x^{2} \right)^{3/2}}$$
%
%
%
%Demuestras las dos primeras f\'ormulas de \eqref{ayr:34.1}.
%
%
%
% Resuelva $$\displaystyle \int \dfrac{dx}{1+\sin(x)-\cos(x)}$$
%
%
%
%Resuelva $$\displaystyle \int \dfrac{dx}{3-2\cos(x)}$$
%
%
%
%Resuelva $$\displaystyle \int \dfrac{dx}{5+4\sin(x)}$$
%
%
%
%Use la sustituci\'on $1-x^{3}=z^{2}$ para resolver 
%$$\displaystyle \int x^{5}\sqrt{1-x^{3}}dx$$
%
%
%
%Use $x=\dfrac{1}{z}$, para resolver $$\displaystyle \int \dfrac{\sqrt{x-x^{2}}}{x^{4}}dx$$
%
%
%
%Resuelva $$\displaystyle \int \dfrac{dx}{x^{1/2}+x^{1/3}}$$
