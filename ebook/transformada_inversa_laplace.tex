\section{Transformada Inversa de Laplace}

%\subsection{Definici\'on}


 \subsection{Definici\'on}
 Una \emph{transformada inversa de Laplace} de $F(s),$ denotada por $\lapin{F(s)},$ es una funci\'on $f(x)$ tal que
 $$
 \lap{f(x)}=F(s).
 $$



 La manera más práctica de encontrar las inversas es identificar, en una tabla de transformadas, la funci\'on $F(s)$ como una transformada de Laplace de una funci\'on $f(x).$
 




	
	Generalmente, esto se hace manipulando algebraicamente $F(s).$


\subsection{Manipulaci\'on de denominadores}


	Para poder encontrar transformadas inversas de Laplace, necesitaremos manipular expresiones algebraicas.
	

{Métodos algebraicos}
	Especialmente, necesitaremos dos técnicas:
	\begin{itemize}
		\item Complemento de cuadrados. 
		\item Fracciones parciales. 
	\end{itemize}

	
{Complemento de cuadrados}
Si $p(x)=ax^2+bx+c,$ entonces
$$
p(x)=a\left( x-h \right)^2+k,
$$ donde $h=-\dfrac{b}{2a}$ y $k=p(h).$


{Fracciones parciales}
Otro m\'etodo útil que se recomienda repasar es la
 \href{http://mathworld.wolfram.com/PartialFractionDecomposition.html}{descomposici\'on en fracciones parciales.}


{Linealidad}

 
 \begin{proposicion}
  \label{bron:prop:22.1}
  Si $\lap{f(x)}=F(s),$ $\lap{g(x)}=G(s)$ y $c_{1},c_{2}\in \R,$ entonces
  $$
  \lapin{c_{1}F(s)+c_{2}G(s)}=c_{1}f(x)+c_{2}g(x).
  $$
 \end{proposicion}



\subsection{Ejemplos}


 \begin{problema}
  \label{bron:exmp:22.1}
  Encontrar $\lapin{\dfrac{1}{s}}$
 \end{problema}



 \begin{problema}
  \label{bron:exmp:22.2}
  Encontrar $\lapin{\dfrac{1}{s-8}}$
 \end{problema}




 \begin{problema}
  \label{bron:exmp:22.3}
  Encontrar $\lapin{\dfrac{s}{s^{2}+6}}$
 \end{problema}




 \begin{problema}
  \label{bron:exmp:22.4}
  Encontrar $\lapin{\dfrac{5s}{\left( s^{2}+1 \right)^{2}}}$
 \end{problema}



%
% \begin{problema}
%  \label{bron:exmp:22.5}
%  Encontrar $\lapin{\dfrac{1}{\sqrt{s}}}$
% \end{problema}
%
%


 \begin{problema}
  \label{bron:exmp:22.6}
  Encontrar $\lapin{\dfrac{s+1}{s^{2}-9}}$
 \end{problema}



% \section{Complemento de cuadrados}
% 
% 
%  Cualquier funci\'on cuadrática se puede reescribir en la forma
%  $$
%  f(x)=a(x-h)^2+k,
%  $$
%  por el m\'etodo de \emph{complementos de cuadrado.}
% 
% 
% 
%   El punto $(h,k)$ se llama \emph{v\'ertice,} y corresponde al \emph{extremo} de la parábola
%   $$
%   y=a(x-h)^2+k.
%   $$
% 
% 
% 
%  La f\'ormula para encontrar el v\'ertice de la parábola
%  $y=f(x)=ax^2+bx+c$ es
%  $$\begin{cases}
%     h=-\dfrac{b}{2a}\\
%     k=f(h).
%    \end{cases}
% $$
% 
% 
% 
%  Para completar el cuadrado, podemos usar el \emph{m\'etodo de divisi\'on sint\'etica:}
%  \begin{center}
% \begin{tabular}{l|lll}
% $h$ & $a$ & $b$ & $c$\\
%   & $\downarrow$ & $+ah$ & $\dots$\\\hline
%   & $a$ & $\dots$ & $k$
%  \end{tabular}
%  \end{center}
% 
% 
% \subsection{Ejemplos}
% 
%   \begin{problema}
%       Complete el cuadrado de
%  $$y=x^2-4x+7.$$
%   \end{problema}
% 
% 
% 
% 
%  \begin{problema}
%   Complete el cuadrádo de
%   $$
%   y=3x^2+30x+63.
%   $$
%  \end{problema}
% 
% 
% 
% 
% 
% 
%  \begin{problema}
%   \label{bron:exmp:22.7}
%   Encuentre
%   $$
%   \lapin{\dfrac{s}{(s-2)^{2}+9}}
%   $$
%  \end{problema}
% 
% 
% 
% 
%  \begin{problema}
%   \label{bron:exmp:22.8}
%   Encuentre
%   $$
%   \lapin{\dfrac{1}{s^{2}-2s+9}}
%   $$
%  \end{problema}
% 
% 
% 
% 
%  \begin{problema}
%   \label{bron:exmp:22.9}
%   Encuentre
%   $$\lapin{\dfrac{s+4}{s^{2}+4s+8}}$$
%  \end{problema}
% 
% 
% 
%  \section{Fracciones parciales}
% 
% 
% {Ejemplo}
% \begin{align*}
%  \dfrac{1}{x-1}+\dfrac{1}{2x+1}
%  &= \dfrac{(2x+1)+(x-1)}{(x-1)(2x+1)} \\
%  &= \dfrac{3x}{2x^{2}-x-1}
% \end{align*}
% 
% {}
% El proceso inverso se conoce como \emph{fracciones parciales}.
% 
% {}
% En esta sección supondremos que $$r(x)=\dfrac{P(x)}{Q(x)}$$ es una función racional, con $P,Q$ polinomios tales que el grado de $P$ es estrictamente menor que el de $Q$.
% 
% 
% {Caso 1}
% Supongamos que podemos factorizar
% \begin{align*}
%  Q(x) = \left( a_{1}x+b_{1} \right)...\left( a_{n}x+b_{n} \right)
% \end{align*}
% sin factores repetidos. En este caso,
% \begin{align*}
%  \dfrac{P(x)}{Q(x)} =
%  \dfrac{A_{1}}{a_{1}x+b_{1}}+...+\dfrac{A_{n}}{a_{n}x+b_{n}}.
% \end{align*}
% 
% {}
% \begin{problema}
%  \label{pre:exmp:9.8.1}
% Encuentre la descomposición en fracciones parciales de
% \begin{align*}
%  \dfrac{5x+7}{x^{3}+2x^{2}-x-2}.
% \end{align*}
% \end{problema}
% 
% 
% {Caso 2}
% Suponga que la factorización de $Q(x)$ contiene el factor lineal $ax+b$ repetido $k$ veces, es decir, $(ax+b)^{k}$ es un factor de $Q(x)$.
% 
% Entonces, en el caso de tal factor, la descomposición en fracciones parciales de $P(x)/Q(x)$ contiene
% \begin{align*}
%  \dfrac{A_{1}}{ax+b}+\dfrac{A_{2}}{\left( ax+b \right)^{2}}+...+\dfrac{A_{k}}{\left( ax+b \right)^{k}}
% \end{align*}
% 
% {}
% \begin{problema}
%  \label{pre:exmp:9.8.2}
%  Encuentre la descomposición en fracciones parciales de
%  \begin{align*}
%  \dfrac{x^{2}+1}{x\left( x-1
%  \right)^{3}}
% \end{align*}
% \end{problema}
% 
% 
% {Caso 3}
% Supongamos que la factorización completa de $Q(x)$ contiene factores cuadráticos irreducibles $ax^{2}+bx+c$. Entonces, en el caso de tal facto, la descomposición en fracciones parciales de $P(x)/Q(x)$ tiene un termino de la forma
% \begin{align*}
%  \dfrac{Ax+B}{ax^{2}+bx+c}.
% \end{align*}
% 
% {}
% \begin{problema}
%  \label{pre:exmp:9.8.3}
%  Encuentre la descomposición en fracciones parciales de
%  \begin{align*}
%  \dfrac{2x^2-x+4}{x^{3}+4x}.
% \end{align*}
% \end{problema}
% 
% 
% {Caso 4}
% Supongamos que la factorización de $Q(x)$ contiene el factor $\left( ax^{2}+bx+c \right)^{k},$ con $ax^{2}+bx+c$ irreducible.
% 
% Entonces, la descomposición en fracciones parciales contiene los términos correspondientes
% \begin{align*}
%  \dfrac{A_{1}x+B_{1}}{ax^{2}+bx+c}+...
%  \dfrac{A_{2}x+B_{2}}{\left( ax^{2}+bx+c \right)^{2}}+...
%  \dfrac{A_{k}x+B_{k}}{\left( ax^{2}+bx+c \right)^{k}}
% \end{align*}
% 
% {}
% \begin{problema}
%  \label{pre:exmp:9.8.4}
%  Escriba la descomposición en fracciones parciales de
%  \begin{align*}
%  \frac{x^{5}-3x^{2}+12x-1}{x^{3}\left( x^{2}+x+1 \right)\left( x^{2}+3 \right)^{3}}
% \end{align*}
% sin determinar los respectivos coeficientes.
% \end{problema}
% 
% 
% {}
% En el caso de que el grado de $P(x)$ sea mayor que el de $Q(x)$, es necesario primero utilizar la división larga de polinomios.
% 
% {}
% \begin{problema}
%  \label{pre:exmp:9.8.5}
%  Encuentre la descomposición en fracciones parciales de
%  \begin{align*}
%  \dfrac{2x^{4}+4x^{3}-2x^{2}+x+7}{x^{3}+2x^{2}-x-2}
% \end{align*}
% sin determinar los coeficientes.
% \end{problema}
% 
% 
% \subsection{Ejemplos}
% 
% % 
% %  \begin{problema}
% %   \label{bron:exmp:22.12}
% %   Descomponga en fracciones parciales
% %   $$
% %   \dfrac{1}{\left( s^{2}+1 \right)\left( s^{2}+4s+8 \right)}
% %   $$
% %  \end{problema}
% %
% % 
% 
% 
%  \begin{problema}
%   \label{bron:exmp:22.13}
%   Descomponga en fracciones parciales
%   $$
%   \dfrac{s+3}{(s-2)(s+1)}
%   $$
%  \end{problema}
% 
% 
% 
% 
% 
%  \begin{problema}
%   \label{bron:exmp:22.15}
%   Usando el resultado del ejemplo \ref{bron:exmp:22.13}, encuentre
%   $$\lapin{\dfrac{s+3}{(s-2)(s+1)}}.$$
%  \end{problema}
% 
% 
% 
%  \begin{problema}
%   \label{bron:exmp:22.14}
%   Descomponga en fracciones parciales
%   $$
%   \dfrac{8}{s^{3}\left( s^{2}-s-2 \right)}
%   $$
%  \end{problema}
% 
% 
% 
% 
%  \begin{problema}
%   \label{bron:exmp:22.16}
%   Usando el resultado del ejemplo \ref{bron:exmp:22.14}, encuentre
%   $$\lapin{\dfrac{8}{s^{3}(s^{2}-s-2)}}.$$
%  \end{problema}
% 
% 
% 
%  \begin{problema}
%   \label{bron:exmp:22.11}
%   Descomponga en fracciones parciales
%   $$
%   \dfrac{1}{(s+1)(s^{2}+1)}
%   $$
%  \end{problema}
% 
% 
% 
% 
%  \begin{problema}
%   \label{bron:exmp:22.17}
%   Usando el resultado del ejemplo \ref{bron:exmp:22.11}, encuentre
%   $$\lapin{\dfrac{1}{(s+1)(s^2+1)}}.$$
%  \end{problema}
% 
% 
% 


