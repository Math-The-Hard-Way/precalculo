\section{Técnicas de integaci\'on trigonométrica}

\subsection{Integrados trigonométricos}


\subsection{Caso 1}
 Considérense las integrales de la forma 
 $$
 \int \sin^{k}(x)cos^{n}(x)dx, 
 $$ con $k,n$ enteros no negativos.



\subsection{Tipo 1.1}
 Si Al menos uno de los números $k,n$ es impar, entonces podemos escoger $u=\cos(x)$ o $u=\sin(x).$



 \begin{problema}
  \label{ayr:32.1}
  $$
  \int \sin^{3}(x)\cos^{2}(x)dx.
  $$
 \end{problema}




 \begin{problema}
 \label{ayr:32.2}
  $$
  \int sin^{4}(x)cos^{7}(x)dx
  $$
 \end{problema}




 \begin{problema}
  \label{ayr:32.3}
  $$
  \int sin^{5}(x)dx
  $$
 \end{problema}




\subsection{Tipo 1.2}
 Si ambas potencias $k,n$ son pares. Entonces utilizaremos las identidades
 $$\begin{cases}
    cos^{2}(x)=\dfrac{1+cos(2x)}{2}\\
    sin^{2}(x)=\dfrac{1-cos(2x)}{2}
   \end{cases}
$$



 \begin{problema}
  \label{ayr:32.4}
  $$
  \int cos^{2}(x)sin^{4}(x)dx
  $$
 \end{problema}




\subsection{Caso 2}
 Consideraremos integrales de la forma 
 $$
 \int tan^{k}(x)sec^{n}(x)dx.
 $$
  y utilizaremos la identidad
$$
sec^{2}(x)=1+tan^{2}(x).
$$
 
 
 
 \subsection{Tipo 2.1} Si $n$ es par, entonces se sustituye $u=\tan(x).$
 

 
 
  \begin{problema}
   \label{ayr:32.5}
   $$
   \int tan^{2}(x)sec^{4}(x)dx
   $$
  \end{problema}

 

 
 \subsection{Tipo 2.2} Si $n,k$ son impares, se sustituye $u=sec(x).$
 

 

 \begin{problema}
  \label{ayr:32.6}
  $$
  \int tan^{3}(x)sec(x)dx.
  $$
 \end{problema}




\subsection{Tipo 2.3}
Si $n$ es impar y $k$ par, reducimos a los casos anteriores y utilizaremos la f\'ormula
$$
\displaystyle \int \sec(x)dx= \ln\abs{\tan(x)+\sec(x)}
$$



\begin{problema}
 $$
 \int \tan^{2}(x)\sec(x)dx=
 $$
\end{problema}



\subsection{Caso 3} Consideremos ahora integrales de la forma $\int f(Ax)g(Bx)dx,$ donde $f,g$ pueden ser o bien $sin$ o bien $cos.$ 




 Necesitaremos las identidades
 \begin{align*}
  sin(Ax)cos(Bx)&=\dfrac{1}{2}\left( sin\left( (A+B)x \right)+sin\left( (A-B)x \right) \right)\\
  sin(Ax)sin(Bx)&=\dfrac{1}{2}\left( cos\left( (A-B)x \right)-cos\left( (A+B)x \right) \right)\\
  cos(Ax)cos(Bx)&=\dfrac{1}{2}\left( cos\left( (A-B)x \right)+cos\left( (A+B)x \right) \right)
 \end{align*}



 \begin{problema}
  \label{ayr:32.7}
  $$
  \int sin(7x)cos(3x)
  $$
 \end{problema}




 \begin{problema}
  \label{ayr:exmp:32.8}
  $$
  \int sin(7x)cos(3x)
  $$
 \end{problema}




 \begin{problema}
  \label{ayr:32.9}
  $$
  \int sin(7x)sin(3x)
  $$
 \end{problema}




 \begin{problema}
  \label{ayr:32.10}
  $$
  \int cos(7x)cos(3x)
  $$
 \end{problema}



\subsection{Sustituci\'on trigonométrica}


Existen tres principales tipos de sustituci\'on trigonométrica. Introduciremos cada uno por medio de ejemplos típicos.



\begin{problema}
 \label{ayr:exmp:32.11}
 Encuentre
 $$
 \displaystyle \int \dfrac{dx}{x^{2}\sqrt{4+x^{2}}}
 $$
\end{problema}



% 
% \subsection{Estrategia I}
%  \begin{center}
% % 
% Si $\sqrt{a^{2}+x^{2}}$ aparece en el integrando, intente $x=a\tan(\theta)$
% \end{center}
%  
% 

%[c]
%\begin{center}
 \subsection{Estrategia I}
  
 Si $\sqrt{a^{2}+x^{2}}$ aparece en el integrando, intente $x=a\tan(\theta)$

%\end{center}



\begin{problema}
 \label{ayr:exmp:32.12}
 Encuentre $$
 \displaystyle \int \dfrac{dx}{x^{2}\sqrt{9-x^{2}}}
 $$
\end{problema}



%[c]
\subsection{Estrategia II}
 
 Si $\sqrt{a^{2}-x^{2}}$ aparece en un integrando, trate con la sustituci\'on $x=a\sin(\theta).$





\begin{problema}
\label{ayr:exmp:32.13}
Encuentre 
$$
\displaystyle \int \dfrac{x^{2}}{\sqrt{x^{2}-4}}dx.
$$
\end{problema}



\subsection{Estrategia III}
  Si $\sqrt{x^{2}-a^{2}}$ aparece en un integrando, trate con la sustituci\'on $x=a\sec{\theta}.$


