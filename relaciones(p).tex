\section*{Problemas}

\begin{problema}
Sea $A=\set{1,2,3,4},$ $B=\set{a,b,c,d}$ y $C=\set{x,y,z}$ y definimos las relaciones: 
$$R=\set{(1,a),(2,d),(3,a),(3,b),(3,d)}$$  $$S=\set{(b,x),(b,z),(c,y),(d,z)}.$$ Encuentre $RS.$
\end{problema}

	\begin{problema}
	Determine cuando una relación $R$ es \emph{no-reflexiva}.
\end{problema}

\begin{problema}
	\label{lip:exmp:2.5}
	Sea $A=\set{1,2,3,4}.$ Determine cuales de las siguientes relaciones son reflexivas:
	\begin{itemize}
		\item $R_{1}=\set{(1,1),(1,2),(2,3),(1,3),(4,4)}$ 
		\item $R_{2}=\set{(1,1),(1,2),(2,1),(2,2),(3,3),(4,4)}$ 
		\item $R_{3}=\set{(1,3),(2,1)}$
		\item $R_{4}=\emptyset$
		\item $R_{5}=A \times A$
	\end{itemize}
	
\end{problema}




\begin{problema}
	\label{lip:exmp:2.6}
	Determine cuales de las siguientes relaciones son reflexivas:
	\begin{itemize}
		\item $\leq$ en $\Z$ 
		\item $\subset$ en $2^{A}$ 
		Aquí $A$ es un conjunto y $2^{A}$ es la colección de todos sus subconjuntos (incluyendo tanto a $\emptyset$ como $A$) 
		\item $\perp$ en el conjunto $L$ de líneas en el plano 
		\item $\parallel$ en el conjunto $L$ de líneas en el plano 
		\item $\mid$ (divisivilidad) en $\N.$  Aquí $a\mid b$ significa que \emph{a divide a b.} 
	\end{itemize} 
\end{problema}

	
\begin{problema}
	Determine cuando una relación $R$ no es sim\'etrica.
\end{problema}




\begin{problema}
	\label{lip:exmp:2.7}
	\begin{enumerate}
		\item   Determine cuales de las relaciones en el ejemplo \ref{lip:exmp:2.5} son sim\'etricas.
		\item Determine cuales de las relaciones en el ejemplo \ref{lip:exmp:2.6} son sim\'etricas.
	\end{enumerate}
	
\end{problema}


\begin{problema}
	Determina cuando una relación $R$ no es sim\'etrica.
\end{problema} 


\begin{problema}
	\label{lip:exmp:2.8}
	\begin{enumerate}
		\item   Determine cuales de las relaciones en el ejemplo \ref{lip:exmp:2.5} son antisim\'etricas.
		\item Determine cuales de las relaciones en el ejemplo \ref{lip:exmp:2.6} son antisim\'etricas.
	\end{enumerate}
	
\end{problema}


\begin{problema}
	Determina cuando una relación $R$ no es transitiva.
\end{problema}




\begin{problema}
	\label{lip:exmp:2.9}
	\begin{enumerate}
		\item   Determine cuales de las relaciones en el ejemplo \ref{lip:exmp:2.5} son transitivas.
		\item Determine cuales de las relaciones en el ejemplo \ref{lip:exmp:2.6} son transitivas.
	\end{enumerate}
	
\end{problema}



\begin{problema}
	\label{lip:exmp:2.12.b}
	La relación $\subset$ no es una relación de equivalencia. Demuestra que aunque es reflexiva y transitiva,  no es sim\'etrica. 
\end{problema}



\begin{problema}
	Para cada relación, verifique que se trata de una relación de equivalencia, y calcule sus clases de equivalencia.
\end{problema}

\begin{itemize}
	\item $\displaystyle R_{0}= \left[\left[\text{\texttt{a}}, \text{\texttt{a}}\right], \left[\text{\texttt{b}}, \text{\texttt{b}}\right], \left[\text{\texttt{c}}, \text{\texttt{c}}\right]\right] $
	\item $\displaystyle R_{1}= \left[\left[\text{\texttt{a}}, \text{\texttt{a}}\right], \left[\text{\texttt{a}}, \text{\texttt{b}}\right], \left[\text{\texttt{b}}, \text{\texttt{a}}\right], \left[\text{\texttt{b}}, \text{\texttt{b}}\right], \left[\text{\texttt{c}}, \text{\texttt{c}}\right]\right] $
	\item $\displaystyle R_{2}= \left[\left[\text{\texttt{a}}, \text{\texttt{a}}\right], \left[\text{\texttt{a}}, \text{\texttt{c}}\right], \left[\text{\texttt{b}}, \text{\texttt{b}}\right], \left[\text{\texttt{c}}, \text{\texttt{a}}\right], \left[\text{\texttt{c}}, \text{\texttt{c}}\right]\right] $
	\item $\displaystyle R_{3}= \left[\left[\text{\texttt{a}}, \text{\texttt{a}}\right], \left[\text{\texttt{b}}, \text{\texttt{b}}\right], \left[\text{\texttt{b}}, \text{\texttt{c}}\right], \left[\text{\texttt{c}}, \text{\texttt{b}}\right], \left[\text{\texttt{c}}, \text{\texttt{c}}\right]\right] $
	\item $\displaystyle R_{4}= \left[\left[\text{\texttt{a}}, \text{\texttt{a}}\right], \left[\text{\texttt{a}}, \text{\texttt{b}}\right], \left[\text{\texttt{a}}, \text{\texttt{c}}\right], \left[\text{\texttt{b}}, \text{\texttt{a}}\right], \left[\text{\texttt{b}}, \text{\texttt{b}}\right], \left[\text{\texttt{b}}, \text{\texttt{c}}\right], \left[\text{\texttt{c}}, \text{\texttt{a}}\right], \left[\text{\texttt{c}}, \text{\texttt{b}}\right], \left[\text{\texttt{c}}, \text{\texttt{c}}\right]\right] $
	
\end{itemize}





\begin{problema}
	\label{lip:exmp:2.13.b}
	Describa las clases de equivalencia de $\Z \mod 5,$ y verifique que las operaciones
	$$
	[a]+[b]=[a+b], \; [a]\cdot[b]=[a \cdot b]
	$$ están bien definidas.   
\end{problema}





\begin{problema}
	Considere el conjunto $S=\set{(a,b)\in \Z^{2}\mid b\neq0}$ y la siguiente relación en este conjunto
	$
	(a,b)\rel{R}(c,d) \iff ad-bc=0.
	$
	
	\begin{enumerate}
		\item Demuestre que $R$ es una relación de equivalencia.
		\item Demuestre que $[(a,b)]=[(c,d)]$ para todo $n\in\Z, n \neq 0$
		$$
		[(a,b)]=[(n\cdot a, n\cdot b)]
		$$
		\item Demuestre que las operaciones
		$$
		\begin{cases}
			[(a,b)]+[(c,d)]=[(ad+bc,bd)] \\
			[(a,b)]\cdot[(c,d)]=[(a\cdot c, b \cdot d)]
		\end{cases}
		$$ están bien definidas
		\item Denote por $\frac{a}{b}$ la clase de equivalencia $[(a,b)]$ y reescriba los resultados anteriores usando esta notación.
		\item ?`Qu\'e conjunto de números representa el cociente $S/R.$?
	\end{enumerate}
\end{problema}



\begin{problema}
	Para cada una de las siguientes relaciones, verifique que es un orden parcial y dibuje su diagrama de Hasse.
\end{problema}
\begin{itemize}
	\item $\displaystyle R_{1}= \left[\left[\text{\texttt{a}}, \text{\texttt{a}}\right], \left[\text{\texttt{b}}, \text{\texttt{b}}\right], \left[\text{\texttt{c}}, \text{\texttt{c}}\right]\right] $
	\item $\displaystyle R_{2}= \left[\left[\text{\texttt{a}}, \text{\texttt{a}}\right], \left[\text{\texttt{a}}, \text{\texttt{b}}\right], \left[\text{\texttt{b}}, \text{\texttt{b}}\right], \left[\text{\texttt{c}}, \text{\texttt{c}}\right]\right] $
	\item $\displaystyle R_{3}= \left[\left[\text{\texttt{a}}, \text{\texttt{a}}\right], \left[\text{\texttt{a}}, \text{\texttt{c}}\right], \left[\text{\texttt{b}}, \text{\texttt{b}}\right], \left[\text{\texttt{c}}, \text{\texttt{c}}\right]\right] $
	\item $\displaystyle R_{4}= \left[\left[\text{\texttt{a}}, \text{\texttt{a}}\right], \left[\text{\texttt{a}}, \text{\texttt{b}}\right], \left[\text{\texttt{a}}, \text{\texttt{c}}\right], \left[\text{\texttt{b}}, \text{\texttt{b}}\right], \left[\text{\texttt{c}}, \text{\texttt{c}}\right]\right] $
	\item $\displaystyle R_{6}= \left[\left[\text{\texttt{a}}, \text{\texttt{a}}\right], \left[\text{\texttt{b}}, \text{\texttt{b}}\right], \left[\text{\texttt{b}}, \text{\texttt{c}}\right], \left[\text{\texttt{c}}, \text{\texttt{c}}\right]\right] $
	\item $\displaystyle R_{7}= \left[\left[\text{\texttt{a}}, \text{\texttt{a}}\right], \left[\text{\texttt{a}}, \text{\texttt{c}}\right], \left[\text{\texttt{b}}, \text{\texttt{b}}\right], \left[\text{\texttt{b}}, \text{\texttt{c}}\right], \left[\text{\texttt{c}}, \text{\texttt{c}}\right]\right] $
	\item $\displaystyle R_{8}= \left[\left[\text{\texttt{a}}, \text{\texttt{a}}\right], \left[\text{\texttt{a}}, \text{\texttt{b}}\right], \left[\text{\texttt{a}}, \text{\texttt{c}}\right], \left[\text{\texttt{b}}, \text{\texttt{b}}\right], \left[\text{\texttt{b}}, \text{\texttt{c}}\right], \left[\text{\texttt{c}}, \text{\texttt{c}}\right]\right] $
\end{itemize}




\begin{problema}
	\label{lip:exmp:2.14}
	Demuestre para cada par $(S,R),$ el conjunto $S$ es parcialmente ordenado respecto a $R:$
	\begin{enumerate}%
		\item $(2^{A}, \subset).$ %Aquí $A$ denota un conjunto arbitrario y $2^{A}$ la colección de todos sus subconjuntos.
		
		\item $(\R, \leq )$
		\item $(\N, \mid).$  Muestre que esto no es cierto para $(\Z, \mid).$
	\end{enumerate}
	
\end{problema}
